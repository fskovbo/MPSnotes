\chapter{Time Evolution Algorithm}

In order to efficiently perform the time evolution a slightly modified version of the TEBD algorithm was employed. Since the Hamiltonian is changing with every time step, one has to account for the time spent exponentiating the operators when considering the runtime of the algorithm. For a general operator $\hat{W}$ this can be achieved through the series expansion
\begin{equation}
	\exp \left( \hat{W} \right) = \sum_{k = 0}^{\infty} \frac{\hat{W}^k}{k!} = \hat{\mathds{1}} + \hat{W} \Bigl(  \hat{\mathds{1}} + \frac{\hat{W}}{2} \Bigl( \hat{\mathds{1}} + \frac{\hat{W}}{3} \Bigl( \ldots
\end{equation}
The number of terms needed in the expansion to accurately describe the exponentiation depends on the operator. As the cost of exponentiating the hopping terms of the Bose-Hubbard Hamiltonian was relatively high, it was concluded that keeping $J$ fixed and using $U$ as the control parameter was the better option. Hence, only the terms $\frac{U}{2} \hat{n}_i (\hat{n}_i -1)$ had to be exponentiated at every time step, which could be done fairly quickly due the diagonal form of the operator.
Although the Hamiltonian should be considered a single tensor, it can be split into its components using the Suzuki-Trotter expansion. To second order this expansion read
\begin{equation}
	\exp\left(  ( \hat{V} + \hat{W}  ) \delta \right) = \exp\left(  \hat{V} \delta /2  \right) \exp\left(  \hat{W} \delta   \right) \exp\left(  \hat{V} \delta /2  \right) + O(\delta^3) \; . \label{eq:SuzukiTrotter}
\end{equation}
Thus, one can divide the MPO of the Hamiltonian into a sequence of tensors, where
\begin{align}
	\hat{\mathcal{U}}_{J}^{[i,i+1]} (\delta t/2) &= \exp \left( -i J \hat{a}_{i}^{\dag} \hat{a}_{i+1} \frac{\delta t}{2} / \hbar \right) \\
	\hat{\mathcal{U}}_{U}^{[i]} (\delta t) &= \exp \left( -i \frac{U}{2} \hat{n}_i (\hat{n}_i -1) \delta t / \hbar \right) 
\end{align}


\begin{figure}[h!]
	\centering
	\begin{tikzpicture}[inner sep=1mm]
\def \reldist {2};
\def \numb {4};
\def \wid {3.2};
\def \size {1.0};
\def \rad {0.5};

	
\foreach \i in  {1,...,\numb} {
	\node[tensor,minimum width= \size cm,minimum height= \size cm, rounded corners = 0.2cm] (A\i)
	at (\i * \reldist, 0) {$A^{[ \i ]}$};
	\draw[-] (A\i) -- (\i * \reldist , -8.7);	
};

\foreach \i in {1,...,3} {
    \pgfmathtruncatemacro{\iplusone}{\i + 1};
    \draw[-] (A\i) -- (A\iplusone);
};

\foreach \i in {1,...,\numb} {
	\pgfmathtruncatemacro{\j}{\i + 1};
    \node[twositeop, minimum width= \wid cm,minimum height= \size cm,rounded corners = \rad cm] (fop\i)
    at (\reldist*\i + \reldist/2, -1.5*\i)
    {\small $\hat{\mathcal{U}}_{J}^{[ \i , \j ]} (\delta t /2)$};
};
	
	
\foreach \i in  {1,...,\numb} {
	\node[operator,minimum width= \size cm,minimum height= \size cm] (op\i)
	at (\i * \reldist, -1.5*\i -1.5)
	{\small $\hat{\mathcal{U}}_{U}^{[ \i ]}$};	
};


\foreach \i in  {1,...,\numb} {
	\node (dots\i) at (\i * \reldist, -8.9) {$\vdots$};
	\draw[-] (dots\i) -- (\i * \reldist , -16);	
};

\foreach \i in {1,...,\numb} {
	\pgfmathtruncatemacro{\j}{\i + 1};
    \node[twositeop, minimum width= \wid cm,minimum height= \size cm,rounded corners = \rad cm] (sop\i)
    at (\reldist*\i + \reldist/2, 1.5*\i - 16.5)
    {\small $\hat{\mathcal{U}}_{J}^{[ \i , \j ]} (\delta t /2)$};
};



\draw[decoration={calligraphic brace,amplitude=10pt}, decorate, line width=1.25pt, xshift=-4pt, yshift=0pt]
(0,-15.5) -- (0,-0.8) node [black,midway,xshift=-0.6cm] 
{\large $\delta t$};


	\node (dot2) at (\numb * \reldist + 2.4,0) {$\dots$};

	\draw[-] (A\numb) -- (dot2);
	
	
\end{tikzpicture}
	\caption{\textit{Tensor diagram depicting a single time step of the modified TEBD algorithm. The time evolution operator has been subjected to a Suzuki-Trotter expansion as detailed in eq. \eqref{eq:SuzukiTrotter}. The tensors of the upper part of the network are contracted with the MPS while sweeping from left to right, whereas the lower part is applied with a right to left sweep.}}
	\label{fig:ModifiedTEBD}
\end{figure}

	
\begin{figure}
\centering % <-- add this
\begin{subfigure}[b]{0.35\textwidth}
	\caption{}  
  	\begin{tikzpicture}[inner sep=1mm]
\def \reldist {1.5};
\def \numb {4};
\def \wid {2.3};
\def \size {1.0};
\def \hi {0.8};
\def \rad {0.4};
\def \vert {1.35};


\foreach \i in  {1,...,\numb} {
	\node[tensorr,minimum width= \size cm,minimum height= \size cm, rounded corners = 0.2cm] (A\i)
	at (\i * \reldist, 0) {$M^{[ \i ]}$};
	\draw[-] (A\i) -- (\i * \reldist , -2.5*\vert);	
};

\foreach \i in {1,...,3} {
    \pgfmathtruncatemacro{\iplusone}{\i + 1};
    \draw[-] (A\i) -- (A\iplusone);
};

\node[tensorc,minimum width= \size cm,minimum height= \size cm, rounded corners = 0.2cm] (C) at (1 * \reldist, 0) {$M^{[1]}$};


\foreach \i in  {1,...,\numb} {
	\draw[-] (A\i) -- (\i * \reldist , -2.6 *\vert);	
};

\draw[-,line width=0.8mm] (A1) -- (A2);

\foreach \i in  {1,...,\numb} {
	\node[operator,minimum width= \hi cm,minimum height= \hi cm] (op\i)
	at (\i * \reldist, -\vert)
	{\scriptsize $\hat{\mathcal{U}}_{U}^{[ \i ]}$};	
};

\foreach \i in {1,3} {
	\pgfmathtruncatemacro{\j}{\i + 1};
    \node[twositeop, minimum width= \wid cm,minimum height= \hi cm,rounded corners = \rad cm] (top\i)
    at (\reldist*\i + \reldist/2, -2*\vert)
    {\small $\hat{\mathcal{U}}_{J}^{[ \i , \j ]} $};
};


\node (dot2) at (\numb * \reldist + 1.4,0) {$\dots$};
\draw[-] (A\numb) -- (dot2);
	
	
\end{tikzpicture}
\end{subfigure}
\begin{subfigure}[b]{0.35\textwidth}
	\caption{}    
  	\begin{tikzpicture}[inner sep=1mm]
\def \reldist {1.5};
\def \numb {4};
\def \wid {2.3};
\def \size {1.0};
\def \hi {0.8};
\def \rad {0.4};
\def \vert {1.35};


\foreach \i in  {1,...,\numb} {
	\node[operator] (A\i)
	at (\i * \reldist, 0) {};
	\draw[-] (A\i) -- (\i * \reldist , -2.5*\vert);	
};

\foreach \i in {1,...,3} {
    \pgfmathtruncatemacro{\iplusone}{\i + 1};
    \draw[-] (A\i) -- (A\iplusone);
};


\foreach \i in  {1,...,\numb} {
	\draw[-] (A\i) -- (\i * \reldist , -2.6 *\vert);	
};

\node[operator] (d1) at (1 * \reldist, -\vert) {};
\node[operator] (d2) at (2 * \reldist, -\vert) {};
\node[operator] (dd1) at (1 * \reldist, -2*\vert) {};
\node[operator] (dd2) at (2 * \reldist, -2*\vert) {};


\draw[-,line width=0.8mm] (A1) -- (d1);
\draw[-,line width=0.8mm] (A2) -- (d2);
\draw[-,line width=0.8mm] (dd1) -- (d1);
\draw[-,line width=0.8mm] (dd2) -- (d2);

\foreach \i in  {1,...,\numb} {
	\node[operator,minimum width= \hi cm,minimum height= \hi cm] (op\i)
	at (\i * \reldist, -\vert)
	{\scriptsize $\hat{\mathcal{U}}_{U}^{[ \i ]}$};	
};

\foreach \i in {1,3} {
	\pgfmathtruncatemacro{\j}{\i + 1};
    \node[twositeop, minimum width= \wid cm,minimum height= \hi cm,rounded corners = \rad cm] (top\i)
    at (\reldist*\i + \reldist/2, -2*\vert)
    {\small $\hat{\mathcal{U}}_{J}^{[ \i , \j ]} $};
};


\node (dot2) at (\numb * \reldist + 1.4,0) {$\dots$};
\draw[-] (A\numb) -- (dot2);

\foreach \i in  {3,...,\numb} {
	\node[tensorr,minimum width= \size cm,minimum height= \size cm, rounded corners = 0.2cm] (B\i)
	at (\i * \reldist, 0) {$M^{[ \i ]}$};
};
\node[tensor,minimum width= \wid cm,minimum height= \size cm, rounded corners = 0.2cm] (AA) at (\reldist*1 + \reldist/2, 0) {\Large $\Theta$};



	\node (node1) at (\reldist +0.2, -0.5*\vert -0.05) {\small \textbf{1}};
	\node (node2) at (2*\reldist +0.2, -0.5*\vert -0.05) {\textbf{1}};
	\node (node3) at (\reldist +0.2, -1.5*\vert) {\textbf{2}};
	\node (node4) at (2*\reldist +0.2, -1.5*\vert) {\textbf{2}};	
	
\end{tikzpicture}
\end{subfigure}
\\ % <-- add this
\vspace{10mm}
\begin{subfigure}[b]{0.35\textwidth}
	\caption{}    	
  	\input{Diagrams/TrotterStep4.tex}
\end{subfigure}
\begin{subfigure}[b]{0.35\textwidth}
	\caption{}  
  	\begin{tikzpicture}[inner sep=1mm]
\def \reldist {2};
\def \numb {2};
\def \wid {3.2};
\def \size {1.0};
\def \rad {0.5};



\node[tensorl,minimum width= \size cm,minimum height= \size cm, rounded corners = 0.2cm] (A1) at (1 * \reldist, 0) {$A^{[1]}$};

\node[tensorc,minimum width= \size cm,minimum height= \size cm, rounded corners = 0.2cm] (A2) at (2 * \reldist, 0) {$A^{[2]}$};


\foreach \i in  {1,...,\numb} {
	\draw[-] (A\i) -- (\i * \reldist , -2.5);	
};

\draw[-] (A1) -- (A2);    

\node[twositeop, minimum width= \wid cm,minimum height= \size cm,rounded corners = \rad cm] (fop2)
    at (\reldist*2 + \reldist/2, -1.5 )
    {$\hat{\mathcal{U}}_{J}^{[2,3]}(\delta t /2)$};


\node (dot2) at (\numb * \reldist + 1.4,0) {$\dots$};
\draw[-,line width=0.8mm] (A\numb) -- (dot2);
	
	
\end{tikzpicture}
\end{subfigure}
\caption{\textit{Sequence of contractions for modified TEBD algorithm. Step \textbf{(i)}: MPS centred on site 1, tensors $A^{[1]}$ and $A^{[2]}$ are contracted. Step \textbf{(ii)}: Two-site tensor is contracted with two-site operator, followed by a splitting of the resulting tensor through an SVD in step \textbf{(iii)}. Lastly, in step \textbf{(iv)}, the center (and thereby the normalisation) is pushed to the next site leaving $A^{[1]}$ left-orthogonalised.}}
\end{figure}

\begin{figure}[h]
	\centering
	\includegraphics[width=0.9\textwidth]{Figures/TimeEvolve1.pdf}
	\caption{\textit{}}
	\label{fig:TimeEvolve1}
\end{figure}

\begin{figure}[h]
	\centering
	\includegraphics[width=0.9\textwidth]{Figures/TimeEvolve2.pdf}
	\caption{\textit{}}
	\label{fig:TimeEvolve2}
\end{figure}

\begin{figure}[h]
	\centering
	\includegraphics[width=0.9\textwidth]{Figures/CompareCutoffs1.pdf}
	\caption{\textit{}}
	\label{fig:Cutoff1}
\end{figure}

\begin{figure}[h]
	\centering
	\includegraphics[width=0.9\textwidth]{Figures/CompareCutoffs2.pdf}
	\caption{\textit{}}
	\label{fig:Cutoff2}
\end{figure}

\begin{figure}[h]
	\centering
	\includegraphics[width=0.9\textwidth]{Figures/CompareTimeStep.pdf}
	\caption{\textit{}}
	\label{fig:TimeStep}
\end{figure}

\end{document}