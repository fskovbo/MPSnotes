\chapter{Introduction}

Ultracold atoms have opened up the possibility of engineering systems, which allow scientists to probe the very fundamentals of quantum mechanics. When reaching temperature scales of nanokelvin, the influence of thermal effects no longer destroy the very fragile systems. Thus, one is capable of directly observing quantum effects. At such temperatures many-body phenomena such as Bose-Einstein condensates takes place, where the ground state of a system gains a macroscopic populations. This results in various properties such as long range coherence and superfluidity. CITE
Ever since the first realisation of a Bose-Einstein condensate CITE, the special properties of this macroscopic quantum state have been used in a wide range of experiments. One method of utilizing Bose-Einstein condensates is to load them into an optical lattice, which is an array of potentials created through the interaction between the cold atoms and laser beams. FIND CITES: can be used for quantum simulations (simulation of magnets, artificial gauge fields), quantum information (spin chains, rydberg gates ).

This note presents the basic theory of quantum many-body systems in optical lattices.
\newpage