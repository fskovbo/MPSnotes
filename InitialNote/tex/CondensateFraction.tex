\chapter{Condensate Fraction}

Consider the Bose-Hubbard Hamiltonian
\begin{equation}
	\hat{H} = -J \sum_{\braket{i,j}} \hat{a}_{i}^{\dag} \hat{a}_j + \frac{U}{2} \sum_{i} \hat{n}_i \left( \hat{n}_i -1 \right) \; , 
\end{equation}
which describes nearest-neighbour interactions in a one-dimensional lattice. This Hamiltonian supports two distinct quantum phases: The Superfluid (SF) phase, $U/J \ll 1$, which exhibit long range correlations, and the Mott-Insulator (MI) phase, $U/J \gg 1$, where sites are isolated, hence no correlation between sites is present. The SF phase is a Bose-Einstein condensate with special attributes, as it exhibits maximal de-localization of the particles and can be approximated as a coherent state. On the other hand, as the individual sites are completely decoupled in the MI phase, no condensate is present. \\
According to the Penrose-Onsager criterion, a Bose-Einstein condensate, and by extension the SF state, is present if and only if the largest eigenvalue, $\lambda_1$, of the single-particle density matrix, $\rho^{(1)}$, is macroscopic
\begin{equation}
	f_c = \frac{\lambda_1}{N_{\mathrm{particles}}} > 0 \; ,
\end{equation} 
where $f_c$ is the \textit{condensate fraction} and $N_{\mathrm{particles}}$ is the total number of particles. Thus, the condensate fraction can be used to determine which state is the dominant of the system, as
\begin{align}
	\lim_{N_{\mathrm{particles}} \to \infty} f_{c}^{\mathrm{SF}} &\to 1 \label{eq:SF_lim} \\
	\lim_{N_{\mathrm{particles}} \to \infty} f_{c}^{\mathrm{MI}} &\to 0 \; , \label{eq:MI_lim}
\end{align}
where the filling-fraction $n = N_{\mathrm{particles}}/N_{\mathrm{sites}}$ is held constant. While the limit for the SF phase will be reached for any number of particles, the MI phase will only be reached in the limit $N_{\mathrm{particles}} \to \infty$.

\section{Numerical Results}
In order to test the precision of the MPS formalism and the ITensor library, the condensate fraction was calculated for a series of $U/J$ with varying particle number. For this the library's DMRG algorithm was used, where the number of sweeps was selected such that the bond dimension remained the same when adding more sweeps. In this case 5 sweeps with a maximum bond dimension of 200 was sufficient. All calculations were performed with unit occupancy. These results were then compared to a similar calculation using exact diagonalisation.

\begin{figure}[h!]
    \centering
    \begin{subfigure}[t]{0.49\textwidth}
        \includegraphics[width=\textwidth]{Figures/Condfrac_4to10.pdf}
        \caption{\textit{Condensate fraction calculated from an MPS after performing 5 DMRG sweeps.}}
        \label{fig:Condfrac_4to10}
    \end{subfigure}
    ~
    \begin{subfigure}[t]{0.49\textwidth}
        \includegraphics[width=\textwidth]{Figures/Confrac_exactvsMPS.pdf}
        \caption{\textit{Absolute difference between the condensate fraction calculated using MPS and using exact diagonalisation.}}
        \label{fig:Condfrac_exactvsMPS}
    \end{subfigure}    
\end{figure}
Figure \ref{fig:Condfrac_4to10} shows the condensate fraction for various $U/J$ calculated using MPS. In the limit $U/J = 0$ the condensate fraction is 1 as expected, thus confirming that the system is in the SF state. Meanwhile, in the limit $U/J = 50$ the condensate fraction is dependent on the number of particles, however, this is just as equation \ref{eq:MI_lim} describes.\\
Comparing the results of the MPS calculation with exact diagonalisation yields figure \ref{fig:Condfrac_exactvsMPS}, which displays the absolute difference between the two calculations. It is clear, that the two approaches yields very similar results for low particle-number.\\
One of the great advantages of MPS is its ability to accurately describe states with a large number of particle, which is outside the scope of exact diagonalisation. 
\begin{figure}[h!]
	\centering
	\includegraphics[width=0.9\textwidth]{Figures/Condfrac_4to50.pdf}
	\caption{\textit{Condensate fraction calculated from an MPS after performing 5 DMRG sweeps.}}
	\label{fig:Condfrac_4to50}
\end{figure}
Figure \ref{fig:Condfrac_4to50} shows the results of the DMRG calculations for up to 50 particles. The condensate fraction behaves as expected in the $U/J \gg 1$ limit, as it tends towards zero for larger particle numbers. However, in the SF limit the results are not at all as described by equation \ref{eq:SF_lim}. This is due to a flaw of the MPS description, as correlation functions of an MPS \textit{always} will decay exponentially, thus limiting the correlation length $\xi$. As the systems grows in size, its length exceeds the correlation length of the MPS. This is an issue when describing a pure SF state, as it exhibits very long range correlations. Thus, the MPS is unable to describe the long-range correlation functions of the SF state, resulting in the abnormal behaviour of the result in the $U/J \ll 1$ limit.

\subsection{Visualization of density matrices}
One way of displaying the correlations of the system is to plot its density matrix, whose entries are given by
\begin{equation}
	\rho_{i,j} = \bra{\psi} \hat{a}_{i}^{\dag} \hat{a}_j \ket{\psi} \; .
\end{equation}
\begin{figure}[h!]
    \centering
    \begin{subfigure}[t]{0.49\textwidth}
        \includegraphics[width=\textwidth]{Figures/DensityMatSF.pdf}
        \caption{\textit{Norm value of density matrix entries for $U/J = 0.1$.}}
        \label{fig:DensityMatSF}
    \end{subfigure}
    ~
    \begin{subfigure}[t]{0.49\textwidth}
        \includegraphics[width=\textwidth]{Figures/DensityMatMI.pdf}
        \caption{\textit{Norm value of density matrix entries for $U/J = 12$.}}
        \label{fig:DensityMatMI}
    \end{subfigure}    
\end{figure}
Figures \ref{fig:DensityMatSF} and \ref{fig:DensityMatMI} show the density matrix of the 50 particle system plotted for the Superfluid and the Mott-Insulator limit respectively. In the SF limit long-range correlations are present, which is seen by large off-diagonal elements. However, as discussed previously, in the MPS description all correlations decay exponentially, which causes issues for the elements furthest from the diagonal (top-right and bottom-left corner). In the MI limit no interaction takes place between sites, hence the correlation length is zero. This leads to a correlation matrix consisting only of diagonal elements of equal magnitude. Figure \ref{eq:MI_lim} shows some off-diagonal elements of low, but not zero, magnitude, since it is made for $U/J = 12$, hence the system is not a pure Mott-Insulator.