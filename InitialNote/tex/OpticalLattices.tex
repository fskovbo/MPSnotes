\chapter{Optical Lattice Potentials}
Cold atoms can be trapped in potentials generated by their own dipole interaction with light. Thus, superimposing laser beams allows one to create optical lattices in various shapes and forms. Such dipole traps realised using far-detuned light have important properties, as (i) they are capable of trapping neutral atoms, and (ii) the optical excitation from the trap is very low \cite{grimm}.\\
Optical lattices are a central component of many experiments, as they not only trap the atoms, but also determines many properties of the system.

\section{Trapping of Neutral Atoms}
Consider a two level atom in the presence of a time-varying electric field $\boldsymbol{E} = \boldsymbol{\varepsilon} E_0 \exp{ \left( i(\boldsymbol{k} \boldsymbol{r} - \omega_L t) \right)}$, where $E_0$ is its amplitude, $\boldsymbol{\varepsilon}$ is its polarization vector, $\boldsymbol{k}$ is its wave vector and $\omega_L$ is its frequency. The interaction between the atom and the electric field causes a perturbation of the atoms energy levels, otherwise known as the AC Stark-shift. The Hamiltonian describing this interaction is
\begin{equation}
	\hat{H}_{int} = \hat{d} \boldsymbol{E} \; , \label{eq:Hdipint}
\end{equation}
where $\hat{d} = -e \hat{r}$ is the electric dipole operator.\\
Consider equation \eqref{eq:Hdipint} as a perturbation to the Hamiltonian of the atom. To first order the non-degenerate, time-independent perturbation of the energy of state $i$ reads $E_{i}^{(i)} = \bra{i} \hat{H}_{int} \ket{i}$. However, only states of opposite parity will contribute to the matrix elements of $\braket{\hat{H}_{int}}$, as \eqref{eq:Hdipint} is linear in space. Thus, all first-order perturbation terms cancel.\\
To second order the perturbation of state $i$ is given by
\begin{equation}
	E_i^{(2)} = \sum_{j \neq i} \frac{ |\bra{j} \hat{H}_{int}\ket{i}|^2}{\varepsilon_i - \varepsilon_j} \; .
\end{equation}
Here, the states $\ket{i}$ and $\ket{j}$ and their corresponding energies, $\varepsilon_i$ and $\varepsilon_j$, are of the dressed state picture, where one considers the combined system of the atom and the light field \cite{cohen1992atom}. The state $\ket{i}$ represents the atom being in its ground state, while the light field contains $n$ photons. Thus, the energy of the combined state is $\varepsilon_i = n \hbar \omega_L$, when setting the energy of the atomic ground state to zero. Meanwhile, state $\ket{j}$ is when the atom has been excited by absorbing one of the photons of the field. Hence, the energy of this state is $\varepsilon_j = \hbar \omega_0 + (n-1) \hbar \omega_L$. Defining the detuning $\Delta = \omega_L - \omega_0$ allows writing the perturbation in the form
\begin{equation}
	E_{g/e}^{(2)}=\pm  \frac{ |\bra{e}\hat{d}\ket{g}|^2}{\Delta} |E_0|^2,
	\label{2ndpert}
\end{equation}
where the upper sign is assigned to the atomic ground state $\ket{g}$. This can be rewritten in order to reflect properties of the atom and the field, by considering the intensity of the light, $I = \frac{1}{2} \epsilon_0 c |\boldsymbol{E}|^2 $, and the decay rate of the atom, $\Gamma$. Thus, equation \eqref{2ndpert} can be written as \cite{grimm} 
\begin{equation}
	E_{e/g}^{(2)}=\pm \frac{3 \pi c^2}{2 \omega_0} \frac{\Gamma}{\Delta}I
	\label{eq:dipolepot}
\end{equation}
This is the AC-Stark shift, which constitutes the dipole potential. For red detuning ($\Delta < 0$) the ground state will experience a negative shift leading to an attractive potential with depth depending on the intensity of the laser. Similarly, a blue-detuned laser ($\Delta > 0$) will repel the atom. An illustration of this can be seen in figure \ref{fig:ac_stark}.
\begin{figure}[!h]
	\centering
	\includegraphics[width=0.5\columnwidth]{Figures/acstark.JPG} 
	\caption{\textit{Light shifts of a two-level atom. Left-hand side,
		red-detuned light ($\Delta < 0$) shifts the ground state down and the
		excited state up by same amounts. Right-hand side, a spatially
		inhomogeneous field like a Gaussian laser beam produces a
		ground-state potential well, in which an atom can be trapped. Figure and 		caption are adopted from \cite{grimm}.}}
	\label{fig:ac_stark} 
\end{figure}
Since the sign is reversed for the excited state, it is important that the atom remains in the ground state. Thus, one has to minimize the scattering with the optical potential. The scattering rate is given as \cite{grimm}
\begin{equation}
	\Gamma_{sc} = \frac{3 \pi c^2}{2 \hbar \omega_{0}^3} \left( \frac{\Gamma}{\Delta} \right) ^2 I \; .
\end{equation}
As the detuning becomes small, the laser becomes resonant with the atom causing a large increase in scattered photons. Therefore, one has to choose a large detuning in order for the potential to remain conservative. However, this comes at the cost of a weaker potential. To compensate this, a high laser intensity must be used, in order for the potential to reach sufficient depth. In practise there will be a limit to the laser power available, however, for most alkali-metal atoms the detuning is typically chosen to be large compared to the excited-state hyperfine structure splitting, which provides enough depth while being sufficient to suppress scattering events \cite{manybodyBloch}. 


\section{Optical Lattices}

The dipole potential in equation \eqref{eq:dipolepot} scales with the intensity of the laser. Thus, superimposing laser beams allows for creating a multitude of different potentials through the interference patterns of the lasers. A simple standing wave from two counter-propagating light fields will lead to an array of potential wells
\begin{equation}
	V(z) = - V_0 \cos^2{k z } \; ,
	\label{eq:standwave}
\end{equation}
 where $V_0 = | \frac{3 \pi c^2}{2 \omega_{0}^3} \frac{\Gamma}{\Delta} 4 I_0 |$ from equation \eqref{eq:dipolepot}. In practise, a one dimensional lattice like that of equation \eqref{eq:standwave} is created by shining a single laser beam at a mirror, whereby it interferes with itself.
\begin{figure}[!h]
	\centering
	\includegraphics[width=0.8\columnwidth]{Figures/OpticalLattice.pdf} 
	\caption{\textit{\textbf{(a)} Two dimensional optical lattice formed by two mutually orthogonal laser beams. These tubes have a characteristic cigar shape, due to the Gaussian profile of the lasers. \textbf{(b)} Upon using three orthogonal laser beams, the result is a three dimensional lattice reminiscent of a cubic crystal. Figure is adopted from \cite{WideraThesis}.}}
	\label{fig:OpticalLattice} 
\end{figure}
Adding another laser beam in a different direction creates a periodic two dimensional potential. For orthogonal polarization of the two lasers, the resulting potential is purely the sum of the sinusoidal standing wave potential, as no interference term is present \cite{lewenstein}. Note, that the lattice is only well defined for distances much smaller than the waist of the laser beams, as the lattice is only present within the overlap of the two beams. Various shapes of the lattice can be achieved by adjusting the angle between the beams, however, the most common setup is using two orthogonal beam creating a square lattice of one dimensional tubes, as seen in figure \ref{fig:OpticalLattice}.
In order to create a three dimensional lattice, as seen in figure \ref{fig:OpticalLattice}, an additional third perpendicular laser beam is needed. In the center of the trap, the lattice potential is then given by
\begin{equation}
	V(x,y,z) = - V_0 \left( \cos^2{k x } + \cos^2{k y } + \cos^2{k z } \right) \; , \label{eq:3Dlattice}
\end{equation}
for distances much smaller than the beam waist. In addition to the lattice, an external harmonic confinement will be present due to the Gaussian profile of the laser beams \cite{manybodyBloch}.

\section{Band Structure}
Consider a periodic potential as described by equation \eqref{eq:standwave}. \textit{Bloch's Theorem} states that energy eigenstates of a periodic potential with lattice vector $\boldsymbol{R}$ and quasi-momentum $q$ can be written as Bloch waves, which takes the form
\begin{equation}
	\phi_{\boldsymbol{q}}^{(n)}(\boldsymbol{r}) = e^{i \boldsymbol{q} \boldsymbol{r}} u^{(n)}(\boldsymbol{r}) \; ,
\end{equation}
which is a plane wave modulated by a function with the same periodicity as the potential $u^{(n)}(\boldsymbol{r}) = u^{(n)}(\boldsymbol{r} + \boldsymbol{R})$. Furthermore, the Bloch waves are periodic in reciprocal space, such that $\psi_{\boldsymbol{q}}^{(n)}(\boldsymbol{r}) = \phi_{\boldsymbol{q} + \boldsymbol{G}}^{(n)}(\boldsymbol{r})$, where $\boldsymbol{G}$ is a reciprocal lattice vector. \cite{kittel} \\
This leads to an energy spectrum in the shape of bands with the periodicity of the first \textit{Brillouin Zone}. Bands are denoted by the band index $n$, and their shape is determined by both the shape and the depth of the potential. The potential depth is often denoted in units of the recoil energy $E_r = \frac{\hbar ^2 k^2}{2 m}$, where $m$ is the mass of the atom, and $k$ is the photon wave number of the light forming the optical lattice. For $V_0 = 0$, the particles are free, hence the bands will be parabolic. Meanwhile, for $V_0 \rightarrow \infty$, no interaction between different wells of the lattice can take place, as the wavefunctions of the trapped atoms will be confined to their respective well. Thus, the lattice is reduced to an array of independent harmonic oscillators, whereby the bands will appear flat with equal spacing \cite{greiner}. 

\section{Localized States}
For lattice potential depths within $5 E_r \leq V_0 \leq 8 E_r$, the lattice is in the \textit{tight binding limit}. Within this range, the wavefunctions of the trapped atoms will only overlap with other wells in their closest proximity. Thus, interactions between wells are almost purely of nearest neighbour nature. Due to how well localized the wavefunctions are, a basis of Wannier functions is ideal for describing the system. Wannier functions are related to Bloch functions through the Fourier transform \cite{kittel1963}
\begin{equation}
	w^{(n)}(\boldsymbol{r}) = \frac{1}{\sqrt{N_L}} \sum_{q} e^{ -i \boldsymbol{q} \boldsymbol{R} } \phi_{\boldsymbol{q}}^{(n)}(\boldsymbol{r}) \; ,
\end{equation} 
where $N_L$ is the number of primitive cells of the lattice. The Wannier functions are well localized and centred around the lattice at site $\boldsymbol{R}$. In the case of a separable periodic potential, like that of equation \eqref{eq:3Dlattice}, the single-particle problem becomes dimensional \cite{kohn1959analyticWannier}. Lastly, Wannier functions obey the orthonormality relation
\begin{equation}
	\int \mathrm{d^3}r \; \; w^{(n) *}(\boldsymbol{r} - \boldsymbol{R}) w^{(n')}(\boldsymbol{r} - \boldsymbol{R'}) = \delta_{n,n'} \delta_{\boldsymbol{R},\boldsymbol{R}'} \; ,
\end{equation}
thus forming a complete basis \cite{manybodyBloch}. 
\begin{figure}[!h]
	\centering
	\includegraphics[width=0.8\columnwidth]{Figures/WannierPlot3.pdf} 
	\caption{\textit{Two one-dimensional Wannier functions plotted for a lattice potential depth of $V_0 = 8 E_r$.}}
	\label{fig:WannierPlot} 
\end{figure}
Figure \ref{fig:WannierPlot} shows Wannier functions plotted for lattice potential depths within the tight binding limit. As evident from the plot, the functions overlap with only their nearest neighbours. In the case of a more shallow lattice, the functions would extends to further wells, while they will tend towards a Gaussian shape as the lattice depth increases \cite{greiner}.