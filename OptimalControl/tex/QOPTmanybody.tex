\chapter{Applications of QOCT to Many-Body Quantum Systems}

A wide variety of experiments utilizes the properties of the Mott Insulator, which is produced by initially loading a Bose-Einstein condensate into a single trap followed by slowly ramping up a periodic lattice potential. As the lattice depth increases a quantum phase transition between the Superfluid and the Mott Insulator takes place. Thus, this sequence is a necessary first step for many experiments \cite{manybodyBloch}. In order to perform such experiments a very low number of defects in the final state is required, thus people have resorted to adiabatic transitions, where one varies the lattice depth slowly across the critical point such that the system remains in the ground state with sufficiently high probability. Naturally this comes at the cost of the sequence being quite slow - it typically being performed in about a hundred milliseconds \cite{JakschZoller}.\\
Preparing the Mott Insulator state at a faster rate is very desirable, as it allows faster data acquisition as well as lowers the timespan in which coherences and defects can be introduced through interactions with the environment. However, faster transitions can only be achieved non-adiabatically, hence numerical simulations are critical in the investigation due to the exponential growth of the Hilbert space near the critical point \cite{Vidal2003}.\\

Calculating a non-adiabatical ramp of the lattice potential, which reaches a final state with sufficiently low defect, was made possible with the unification of the quantum optimal control theory and the tDMRG algorithm \cite{Doria2011,FrankBloch}. 

\section{Bloch blabla}
