\chapter{Numerical Results}

Figure REF displays Wannier functions calculated using equation \ref{eq:wannierrescaled} for various lattice depths. As expected the functions become more localized for deeper lattices, thus reducing the overlap with functions at neighboring sites. This will result in a reduction of the parameter $J$ (equation \ref{eq:J}), as hopping to neighboring sites become decreasingly popular as the lattice depth increases. At large lattice depths the Wannier functions are well approximated by Gaussians, as the sites will be almost completely decoupled, whereby the eigenstates will be the same as for the harmonic oscillator. 

The above calculation was performed in one dimension, however, when computing parameters for the Bose-Hubbard model one must take into account all three dimensions. This is due to the Wannier functions having a final extension in all dimensions even though the lattice is effectively one dimensional. Hence, for the following calculations a transverse lattice depth of $V_{0,t} = 20 E_r$ was used (same as in \cite{FrankBloch}. This depth effectively decouples neighboring sites in the transverse directions, thus creating a lattice consisting of an array of tubes.
Figure REF displays the evolution of the Bose-Hubbard parameters as the lattice depth is increased along the final dimension. While $J$ decreases as expected, the stronger confinement and following increase in localization of the Wannier functions results in an increase in $U$. In total the fraction $U/J$ increases with the lattice depth, and at POINT the critical point of the quantum phase transition from Superfluid to Mott-Insulator for one dimension is reached \cite{Kuhner2000}. 