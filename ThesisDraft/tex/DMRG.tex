\chapter{Calculating Ground States with MPS}
For large systems exact diagonalization of the Hamiltonian is impossible, whereby one must resort to variational methods in order to find ground states. This involves finding the MPS $\ket{\psi}$, which minimizes
\begin{equation}
	E = \frac{\bra{\psi} \hat{H} \ket{\psi}}{\braket{\psi | \psi}} \; .
\end{equation}
The most efficient way to do this, is to express the Hamiltonian as an MPO as described in equation \ref{eq:MPOrep}, which makes it easy to aplly to an MPS. While this may seem difficult at first, it can actually be done rather easily without any lengthy calculations. To see this consider the following example:

\subsection{Example: Building an MPO from a Hamiltonian}
Consider a Hamiltonian consisting of next-neighbour interaction terms and a field term
\begin{equation}
	\hat{H} = \sum_i \frac{J}{2} \hat{S}_{i}^{+} \hat{S}_{i+1}^{-} + \frac{J}{2} \hat{S}_{i}^{-} \hat{S}_{i+1}^{+} + J^z \hat{S}_{i}^{z} \hat{S}_{i+1}^{z} - h \hat{S}_{i}^{z} \; ,
	\label{eq:exHamiltonian}
\end{equation}
which can be expressed as a sum of strings of operators - most of these being identities just like in equation \ref{eq:localOperator}. Moving through such a string from the right one will at some point encounter one of 4 non-trivial operators. This can be summarized as 5 different possible states of the string of operators on a given bond:
\begin{enumerate}
	\item
	Only identities to the right of the bond.
	\item
	An $\hat{S}^{+}$ operator just to the right of the bond.
	\item
	An $\hat{S}^{-}$ operator just to the right of the bond.
	\item
	An $\hat{S}^{z}$ operator just to the right of the bond.
	\item
	A completed interaction \textit{or} the field term, $h \hat{S}^{z}$, somewhere to the right.
\end{enumerate}
When moving through the string of operators from the right, only certain transitions between these state are possible. For instance $\boldsymbol{1 \rightarrow 1}$, where one starts in state 1, and the next operator is an identity. Likewise, $\boldsymbol{1 \rightarrow 2,3,4,5}$ are all possible, since these transitions represent the next operator being one of the interactions or the field. Next, $\boldsymbol{2 \rightarrow 5}$ completes the interaction $\frac{J}{2} \hat{S}^{+} \hat{S}^{-}$ - similar transitions exists for all the possible interactions $\boldsymbol{3,4 \rightarrow 5}$. Finally, the transition $\boldsymbol{5 \rightarrow 5}$ is needed to continue iterating through the operator chain after having passed the interactions. This can be encoded in the operator-valued matrix
\begin{equation}
 W^{[i]} \: = \: \begin{pmatrix}
\hat{I} & 0 & 0 & 0 & 0 \\
\hat{S}^+ & 0 & 0 & 0 & 0 \\
\hat{S}^- & 0 & 0 & 0 & 0 \\
\hat{S}^z & 0 & 0 & 0 & 0 \\
-h \hat{S}^z & (J/2) \hat{S}^- & (J/2) \hat{S}^+ & J^z \hat{S}^z & \hat{I}

\end{pmatrix} \; ,
\label{eq:MPOmatrix}
\end{equation}
which contains all the allowed transitions between the five state \cite{Schollwock}. When one starts moving through the operator chain from the right side, one obviously begins in state 1 and ends in state 5. This can be encoded in the two vectors
\begin{equation*}
 \vec{v}_{left} = (0 , 0 , 0 ,0 , 1) \quad , \quad \vec{v}_{right} = (1 , 0 , 0 ,0 , 0)^T \; .
\end{equation*}
Thus, the Hamiltonian described in equation \ref{eq:exHamiltonian} can be written as an MPO 
\begin{equation}
	\hat{H} = \sum_{\substack{j_1 , \ldots , j_N  \\ j_1 ' , \ldots , j_N '}} \vec{v}_{left} \; W^{[1] j_1 , j_1 '} W^{[2] j_2 , j_2 '} \ldots W^{[N] j_N , j_N '} \; \vec{v}_{right} \; \ket{j_1 , \ldots , j_N} \bra{j_1 ' , \ldots , j_N '} \; .
	\label{eq:MPOhamiltonian}
\end{equation}
Observe how all of this was done without having to do a single numerical computation. The resulting MPO can now readily be applied to an MPS.\\

This is just an example for a simple Hamiltonian with only next-neighbour interactions. Interactions of longer range require intermediate states consisting of identities in order to connect the interacting operators. However, doing this is fairly simple and will often result in MPO's of a similar low dimension as the one shown in the example.


\section{Efficient Application of a Hamiltonian to an MPS}
In section \ref{sec:MPO} it was explained how to apply an MPO to an MPS, and it was shown how to efficiently evaluate local operators. While Hamiltonians may contains local terms, they are rarely completely local, and since $\hat{H} \ket{\psi}$ must be evaluated many times during variational methods, the naive approach of section \ref{sec:MPO} simply doesn't cut it.\\
Consider an MPS in a mixed-canonical form
\begin{align}
	\ket{\psi} &= \sum_{\boldsymbol{j}} A^{j_1} \ldots A^{j_{n-1}} \Psi^{j_n} B^{j_{n+1}} \ldots B^{j_{N}} \nonumber \\
	&= \sum_{\alpha_{n-1} , \alpha_{n}} \ket{\alpha_{n-1}}_A \Psi_{\alpha_{n-1} , \alpha_{n}}^{j_n}  \ket{\alpha_{n}}_B \; ,
	\label{eq:MPSmixedsingle}
\end{align}
where the set of matrices at the n'th site has not been brought into either canonical form, and a Hamiltonian in the same form as described in equation \ref{eq:MPOhamiltonian}. For notations sake the opening and closing vectors have been combined with the outer matrices of the MPO.
For the variational search of the ground state huge reductions in computational cost can be found by re-using as many calculations as possible. Consider in the basis $\{ \ket{\alpha_{n-1} } \; , \; \ket{ j_n } \; , \; \ket{\alpha_{n} } \}$ the individual matrix elements of the Hamiltonian
\begin{equation*}
	\bra{\alpha_{n-1} j_n \alpha_{n}} \hat{H} \ket{\alpha_{n-1} ' j_n ' \alpha_{n} '} = \sum_{\boldsymbol{j} , \boldsymbol{j'}}  W^{j_1 , j_1 '} \ldots W^{j_N , j_N '}  \braket{\alpha_{n-1} j_n \alpha_{n} | \boldsymbol{j}} \braket{\boldsymbol{j'} | \alpha_{n-1} ' j_n ' \alpha_{n} '} 
\end{equation*}
Since the basis of local states $\{ \ket{\boldsymbol{j}} \}$ shares a state with the basis $\{ \ket{\alpha_{n-1} } \; , \; \ket{ j_n } \; , \; \ket{\alpha_{n} } \}$, the above expression can be re-written using $\sum_{\boldsymbol{j}} \braket{j_n | \boldsymbol{j}} = \sum_{\boldsymbol{j *}} \ket{ \boldsymbol{j *}}$, where "$\boldsymbol{*}$" means "excluding $j_n$". Thus,
\begin{align}
&	\bra{\alpha_{n-1} j_n \alpha_{n}} \hat{H} \ket{\alpha_{n-1} ' j_n ' \alpha_{n} '} \nonumber \\
 &= \sum_{\boldsymbol{j *} , \boldsymbol{j' * }}  W^{j_1 , j_1 '} \ldots W^{j_n , j_n '} \ldots W^{j_N , j_N '} \nonumber \\
	& \qquad \times \braket{\alpha_{n-1} | j_1 \ldots j_{n-1}} \braket{\alpha_{n} | j_{n+1} \ldots j_{N}} \braket{j_1 ' \ldots j_{n-1} ' | \alpha_{n-1} '} \braket{j_{n+1} ' \ldots j_{N} ' | \alpha_{n} '} \nonumber \\
	&= \sum_{\boldsymbol{j *} , \boldsymbol{j' * }}  W^{j_1 , j_1 '} \ldots W^{j_n , j_n '} \ldots W^{j_N , j_N '} \nonumber \\
	& \qquad \times \left( A^{j_1} \ldots A^{j_{n-1}} \right)_{1 , \alpha_{n-1}}^{*} \left( B^{j_{n+1}} \ldots B^{j_{N}} \right)_{ \alpha_{n} , 1}^{*} \left( A^{j_1 '} \ldots A^{j_{n-1} '} \right)_{1 , \alpha_{n-1} '} \left( B^{j_{n+1} '} \ldots B^{j_{N} '} \right)_{ \alpha_{n} ' , 1} \nonumber \\
	&= \sum_{\alpha_n , \beta_n , \alpha_n '}
	\left( \sum_{j_1 , j_1 '} A_{1 , \alpha_1}^{j_1 *} W_{1, \beta_1}^{j_1 , j_1 '} A_{1 , \alpha_1 '}^{j_1 '} \right)
	\left( \sum_{j_2 , j_2 '} A_{\alpha_1 , \alpha_2}^{j_2 *} W_{\beta_1, \beta_2}^{j_2 , j_2 '} A_{\alpha_1 ' , \alpha_2 '}^{j_2 '} \right)
	\ldots W_{\beta_{n_1}, \beta_n}^{j_n , j_n '} \nonumber \\
	& \qquad \times \left( \sum_{j_{n+1} , j_{n+1} '} B_{\alpha_n , \alpha_{n+1}}^{j_{n+1} *} W_{\beta_n, \beta_{n+1}}^{j_{n+1} , j_{n+1} '} B_{\alpha_n ', \alpha_{n+1} '}^{j_{n+1} '} \right)
	\left( \sum_{j_{N} , j_{N} '} B_{\alpha_{N-1} , 1}^{j_{N} *} W_{\beta_{N-1}, 1}^{j_{N} , j_{N} '} B_{\alpha_{N-1}' , 1 }^{j_{N} '} \right) \nonumber \; .
\end{align}  
While this expression may seem terrible complicated due to all the indices, it is actually rather easy to understand. First, the matrix element is written excluding the local basis state $\ket{j_n}$. Next, the Hamilton MPO is projected into the block states of A and B, $\ket{\alpha_{n-1}}_A$ and $\ket{\alpha_{n}}_B$, which can easily be seen when considering equations \ref{eq:mixedA} and \ref{eq:mixedB}. Finally, the matrices are grouped according to their expansion in the local basis. Working with the above expression appears cumbersome, but as can be seen in figure \ref{fig:singleElemHamil}, what has been done is simply a decoupling of the system into three distinct parts.
\begin{figure}[h!]
	\centering
	\begin{tikzpicture}[inner sep=1mm]
    \foreach \i in {1,...,4} {
        \node[tensor] (tt\i) at (2* \i -1, 0) {};
        \node[operator] (o\i) at (2* \i -1, -1.5) {};
        \node[tensor] (tb\i) at (2* \i -1, -3) {};
        
        
        \draw[-] (tt\i) -- (o\i);
        \draw[-] (o\i) -- (tb\i);
    };
    \foreach \i in {6,...,7} {
        \node[tensor] (tt\i) at (2* \i -1, 0) {};
        \node[operator] (o\i) at (2* \i -1, -1.5) {};
        \node[tensor] (tb\i) at (2* \i -1, -3) {};
        
        
        \draw[-] (tt\i) -- (o\i);
        \draw[-] (o\i) -- (tb\i);
    };
        \foreach \i in {1,...,3} {
        \pgfmathtruncatemacro{\iplusone}{\i + 1};
        \draw[-] (tt\i) -- (tt\iplusone);
        \draw[-] (tb\i) -- (tb\iplusone);
        \draw[-] (o\i) -- (o\iplusone);
    };
    \draw[-] (tt6) -- (tt7);
    \draw[-] (tb6) -- (tb7);
    \draw[-] (o6) -- (o7);
    
    
       
	\node[operator] (o5) at (2* 5 -1, -1.5) {};    
    
    \node (a1) at (8,0.5) {$\alpha_{n-1}$};
    \node (a2) at (10,0.5) {$\alpha_{n}$};
    \node (a1p) at (8,-3.5) {$\alpha_{n-1} '$};
    \node (a2p) at (10,-3.5) {$\alpha_{n} '$};
    \node (j) at (9,-0.5) {$j_n$};
    \node (jp) at (9,-2.5) {$j_n '$};
    
    \node (a1d) at (8,0) {};
    \node (a2d) at (10,0) {};
    \node (a1pd) at (8,-3) {};
    \node (a2pd) at (10,-3) {};
    
    \draw[-] (tt4) -- (a1d);
    \draw[-] (tt6) -- (a2d);
    \draw[-] (tb4) -- (a1pd);
    \draw[-] (tb6) -- (a2pd);
    \draw[-] (o5) -- (j);
    \draw[-] (o5) -- (jp);
    \draw[-] (o5) -- (o4);
    \draw[-] (o5) -- (o6);
    
    \node (d1) at (8,-0.75) {};
    \node (d2) at (8,-2.25) {};
   	\draw[dashed] (d1) -- (d2);
   	\node (d3) at (10,-0.75) {};
    \node (d4) at (10,-2.25) {};
   	\draw[dashed] (d3) -- (d4);
   	
   	
   	\node (L) at (4,-4.5) {L};
   	\node (W) at (9,-4.5) {W};
   	\node (R) at (12,-4.5) {R};
\end{tikzpicture}
	\caption{\textit{Representation of the matrix element $\bra{\alpha_{n-1} j_n \alpha_{n}} \hat{H} \ket{\alpha_{n-1} ' j_n ' \alpha_{n} '}$ as an MPS/MPO network. Expressing the matrix element in this form decouples the network in three parts: The matrices of the MPO, $W^{[n]}$, connecting the physical sites of the matrix element, and contracted parts L and R consisting of the rest of the MPO and thepart of the MPS in left- and right-canonical form respectively.}}
	\label{fig:singleElemHamil}
\end{figure}
Since both the left and right side of the network is connected, one can contract these parts into two separate tensors $L$ and $R$ called "environments":
\begin{align}
	L_{\alpha_{n-1}, \beta_{n-1} , \alpha_{n-1} '} &= \sum_{ \substack{ \{ \alpha_i \beta_i \alpha_i ' \} \\ i < n-1}} \left( \sum_{j_1 , j_1 '} A_{1 , \alpha_1}^{j_1 *} W_{1, \beta_1}^{j_1 , j_1 '} A_{1 , \alpha_1 '}^{j_1 '} \right) \ldots \left( \sum_{j_{n-1} , j_{n-1} '} A_{\alpha_{n-2} , \alpha_{n-1}}^{j_{n-1} *} W_{\beta_{n-2}, \beta_{n-1}}^{j_{n-1} , j_{n-1} '} A_{\alpha_{n-2} ' , \alpha_{n-1} '}^{j_{n-1} '} \right) \label{eq:Ltensor} \\
	R_{\alpha_{n} ,\beta_{n} , \alpha_{n} '} &= \sum_{ \substack{ \{ \alpha_i \beta_i \alpha_i ' \} \\ i > n}} \left( \sum_{j_{n+1} , j_{n+1} '} B_{\alpha_n , \alpha_{n+1}}^{j_{n+1} *} W_{\beta_n, \beta_{n+1}}^{j_{n+1} , j_{n+1} '} B_{\alpha_n ', \alpha_{n+1} '}^{j_{n+1} '} \right) \left( \sum_{j_{N} , j_{N} '} B_{\alpha_{N-1} , 1}^{j_{N} *} W_{\beta_{N-1}, 1}^{j_{N} , j_{N} '} B_{\alpha_{N-1}' , 1 }^{j_{N} '} \right) \label{eq:Rtensor}
\end{align}
From these contractions the obvious tripartite structure of the Hamiltonian matrix element, which could be seen in figure \ref{fig:singleElemHamil}, can be written in a compact way
\begin{equation}
	\bra{\alpha_{n-1} j_n \alpha_{n}} \hat{H} \ket{\alpha_{n-1} ' j_n ' \alpha_{n} '} = \sum_{\beta_{n-1} , \beta_{n}} L_{\alpha_{n-1}, \beta_{n-1} , \alpha_{n-1} '} \; W_{\beta_{n_1}, \beta_n}^{j_n , j_n '} \; R_{\alpha_{n} ,\beta_{n} , \alpha_{n} '} \; .
\end{equation}
Finally, applying this parametrization of the Hamiltonian to the MPS described in equation \ref{eq:MPSmixedsingle} yields \cite{Schollwock}
\begin{equation}
	\hat{H} \ket{\psi} = \sum_{\beta_{n-1} , \beta_{n}} \sum_{\alpha_{n-1}' , j_n ', \alpha_{n}'} L_{\alpha_{n-1}, \beta_{n-1} , \alpha_{n-1} '} \; W_{\beta_{n_1}, \beta_n}^{j_n , j_n '} \; R_{\alpha_{n} ,\beta_{n} , \alpha_{n} '} \; \Psi_{\alpha_{n-1} ' , \alpha_{n} '}^{j_n '} \ket{\alpha_{n-1}}_A \ket{j_n} \ket{\alpha_{n}}_B \; .
	\label{eq:HPsi}
\end{equation}

As mentioned earlier, evaluating $\hat{H} \ket{\psi}$ must be done many times during a variational search of the ground state, hence this operation must be executed as fast as possible. Examining equation \ref{eq:HPsi} one will notice, that while the boundaries of $L$ and $R$ will change depending on which site is being optimized, the bulk of the two tensors remain constant through a lot of the calculations. Instead of calculating $L$ and $R$ from equations \ref{eq:Ltensor} and \ref{eq:Rtensor} explicitly every time equation \ref{eq:HPsi} has to be evaluated, one can iteratively build them, since they only change by one column of the network at a time. This allows reuse of a lot of information resulting in much lower computational cost.\\
Consider the construction of the tensor $L^{[i]}$. This can be built iteratively from the left by contracting the previous left-tensor $L^{[i-1]}$ with the i'th column of the network consisting of $A^{[i]}$, $W^{[i]}$ and $A^{[i] \dag}$
\begin{equation}
	L_{\alpha_i , \beta_i , \alpha_i '}^{[i]} = \sum_{\substack{ j_i , j_i ' \\ \alpha_{i-1} , \beta_{i-1} , \alpha_{i-1} '}} W_{\beta_{i-1} , \beta_i}^{[i] j_i , j_i '} \left( A^{[i] j_i \dag} \right)_{\alpha_i , \alpha_{i-1}} L_{\alpha_{i-1} , \beta_{i-1} , \alpha_{i-1} '}^{[i-1]} A_{\alpha_{i-1} ' , \alpha_i '}^{[i] j_i '} \; .
\end{equation}
This iterative update of $L^{[i]}$ can be seen illustrated in figure \ref{fig:buildLTensor}. The square-bracket notation has been re-introduced to keep track of the tensors relation to the physical sites. In order to remain consistent with notation, the dummy scalars $L_{\alpha_0 , \beta_0 , \alpha_0 '}^{[0]}  = 1  = \alpha_0 , \beta_0 , \alpha_0 '$ has been introduced. 
\begin{figure}[h!]
	\centering
	\begin{tikzpicture}[inner sep=1mm]
    \foreach \i in {1,...,3} {
        \node[tensor] (tt\i) at (\i, 0) {};
        \node[operator] (o\i) at (\i, -1.5) {};
        \node[tensor] (tb\i) at (\i, -3) {};  
        
        \draw[-] (tt\i) -- (o\i);
        \draw[-] (o\i) -- (tb\i);
    };
    \foreach \i in {1,...,2} {
        \pgfmathtruncatemacro{\iplusone}{\i + 1};
        \draw[-] (tt\i) -- (tt\iplusone);
        \draw[-] (tb\i) -- (tb\iplusone);
        \draw[-] (o\i) -- (o\iplusone);
    };
    \foreach \i in {9,...,12} {
        \node[tensor] (tt\i) at (\i, 0) {};
        \node[operator] (o\i) at (\i, -1.5) {};
        \node[tensor] (tb\i) at (\i, -3) {};  
        
        \draw[-] (tt\i) -- (o\i);
        \draw[-] (o\i) -- (tb\i);
    };
    \foreach \i in {9,...,11} {
        \pgfmathtruncatemacro{\iplusone}{\i + 1};
        \draw[-] (tt\i) -- (tt\iplusone);
        \draw[-] (tb\i) -- (tb\iplusone);
        \draw[-] (o\i) -- (o\iplusone);
    };
    
    \node[tensor] (tt4) at (5.4, 0) {};
    \node[operator] (o4) at (5.4, -1.5) {};
    \node[tensor] (tb4) at (5.4, -3) {};

	\draw[-] (tt4) -- (o4);
	\draw[-] (tb4) -- (o4);
    
    \node (at1) at (4.3, 0) {$\alpha_{i-1}$};
    \node (at2) at (6.4, 0) {$\alpha_{i}$};
    \node (ab1) at (4.3, -3) {$\alpha_{i-1} '$};
    \node (ab2) at (6.4, -3) {$\alpha_{i} '$};
    \node (b1) at (4.3, -1.5) {$\beta_{i-1}$};
    \node (b2) at (6.4, -1.5) {$\beta_{i}$};
    
    \draw[-] (tt3) -- (at1);
    \draw[-] (tb3) -- (ab1);
    \draw[-] (o3) -- (b1);
    \draw[-] (tt4) -- (at1);
    \draw[-] (tb4) -- (ab1);
    \draw[-] (o4) -- (b1);
    \draw[-] (tt4) -- (at2);
    \draw[-] (tb4) -- (ab2);
    \draw[-] (o4) -- (b2);
    
    
    \draw[->, line width=1mm] (6.9,-1.5) -- (8.4,-1.5);
    
    \node (at3) at (13, 0) {$\alpha_{i}$};
    \node (ab3) at (13, -3) {$\alpha_{i} '$};
    \node (b3) at (13, -1.5) {$\beta_{i}$};
    \draw[-] (tt12) -- (at3);
    \draw[-] (tb12) -- (ab3);
    \draw[-] (o12) -- (b3);
    
    \node (L1) at (2,-4) {$L^{[i-1]}$};
    \node (L2) at (10.5,-4) {$L^{[i]}$};

    
    \node (eq) at (14,-1.5) {$\equiv$};
    

    \node (dummy1) at (16,0) {$\alpha_{i}$};
 	\node (dummy2) at (16,-3) {$\alpha_{i} '$};
 	\node (dummy3) at (16,-1.5) {$\beta_{i}$};
	\node[tensor] (mat) at (15 , -1.5) {};    
    
    \draw[-] (dummy1.west) .. controls (15, 0) .. (mat.north);
    \draw[-] (dummy2.west) .. controls (15, -3) .. (mat.south);
	\draw[-] (mat) -- (dummy3);   
	
	
	\node (L3) at (15.5,-4) {$L^{[i]}$}; 
\end{tikzpicture}
	\caption{\textit{Iterative update from $L^{[i-1]}$ to $L^{[i]}$. This is done through a contraction of $L^{[i-1]}$ with $A^{[i]}$, $W^{[i]}$ and $A^{[i] \dag}$. The result is a tensor with three horizontal legs.}}
	\label{fig:buildLTensor}
\end{figure}
Performing this update can be sped up when considering the properties of the Hamiltonian and the canonical form. Looking at matrix representation, $W^{[i]}$, of the example Hamiltonian in equation \ref{eq:MPOmatrix} one will notice that most entries are zeroes. While this may not be the case for every Hamiltonian, knowing the structure of the corresponding MPO allows one to simplify multiplications. Furthermore, for the last index in the $W^{[i]}$ matrix, $\beta_i = D_W$, only identities will operate towards the left, if the MPO is constructed according to the rules outlined earlier. Since the matrices $A^{[i]}$ are left normalized, this implies that $L_{\alpha_i , D_W, \alpha_i '}^{[i]} = \delta_{\alpha_i , \alpha_i '}$, greatly simplifying calculations.\\
It is important to store every iteration of $L^{[i]}$, since $L$ will grow and shrink constantly throughout the variational search of the ground state, whereby every iteration of $L^{[i]}$ will be used multiple times. By keeping track of every iteration, an update of the MPS at the $j$'th site will only affect $L^{[i]}$ for $i \geq j$ - and this update can be done swiftly using the previously stored iterations.\\

Naturally all of this applies when building the right tensor $R$. Here one starts from the right and moves left when iteratively contracting the tensor. Likewise, right-normalization simplify the contractions when $b_i = 1$, implying only identities to the left in the chain of operators.\\

Finally, a re-bracketing for optimal matrix multiplication allows for an even lower operational cost
\begin{align*}
	L_{\alpha_i , \beta_i , \alpha_i '}^{[i]} &= \sum_{j_i , \alpha_{i-1}} \left( A^{[i] j_i \dag} \right)_{\alpha_i , \alpha_{i-1}} \left( \sum_{j_i ' , \beta_{i-1}} W_{\beta_{i-1} , \beta_i}^{[i] j_i , j_i '} \left( \sum_{\alpha_{i-1} '}  L_{\alpha_{i-1} , \beta_{i-1} , \alpha_{i-1} '}^{[i-1]} A_{\alpha_{i-1} ' , \alpha_i '}^{[i] j_i '} \right) \right) \\
	 \hat{H} \ket{\psi} &= \sum_{\beta_{n-1} , \alpha_{n-1} '}  L_{\alpha_{n-1}, \beta_{n-1} , \alpha_{n-1} '} \left( \sum_{\beta_n , j_n '} W_{\beta_{n_1}, \beta_n}^{j_n , j_n '} \left( \sum_{\alpha_n '} R_{\alpha_{n} ,\beta_{n} , \alpha_{n} '} \; \Psi_{\alpha_{n-1} ' , \alpha_{n} '}^{j_n '} \right) \right) \ket{\alpha_{n-1}}_A \ket{j_n} \ket{\alpha_{n}}_B \; ,
\end{align*}
where they both scale as $\mathrm{O}(d D^3 D_W)$.\\
  

\section{Iterative Ground State Search}
In order to find the ground state of the system on can introduce a Lagrangian multiplier and extremize
\begin{equation}
	\bra{\psi} \hat{H} \ket{\psi} - \lambda \braket{\psi | \psi} \; ,
	\label{eq:lagrange}
\end{equation}
whereby the desired ground state, $\ket{\psi}$, and ground state energy, $\lambda^0$, will be reached.\\
Consider an MPS of the form $M^{[1]} \ldots M^{[N]}$. Trying to optimize the entire MPS at the same time is a highly non-linear problem involving an extremely large number of variables. However, the problem can be linearised by only considering the matrix entries $M_{\alpha_{n-1}, \alpha_n}^{j_n}$ at site $n$ as variables, while keeping the rest of the MPS constant. By varying the matrix elements one site at a time, one will continuously find states lower in energy until convergence is reached. However, this procedure is very prone to getting stuck in a local extrema. To circumvent this one can consider two sites at a time and optimize with regards to a tensor, created by momentarily merging the two sites.\cite{WhiteDMRG} \\
Considering equation \ref{eq:lagrange} one has to calculate
\begin{align}
	\bra{\psi} \hat{H} \ket{\psi} &= \sum_{\substack{j_n , j_n ' \\ j_{n+1} , j_{n+1} '}} \sum_{\alpha_{n-1} ' , \alpha_n ', \alpha_{n+1} '} \sum_{\alpha_{n-1} , \alpha_n , \alpha_{n+1}} \sum_{\beta_{n-1} , \beta_n , \beta_{n+1}} L_{\alpha_{n-1}, \beta_{n-1} , \alpha_{n-1} '}^{[n-1]} \; W_{\beta_{n_1}, \beta_n}^{[n] j_n , j_n '} \; W_{\beta_{n}, \beta_{n+1}}^{[n+1] j_{n+1} , j_{n+1} '} \nonumber \\
	& \qquad \times R_{\alpha_{n+1} ,\beta_{n+1} , \alpha_{n+1} '}^{[n+2]} \; M_{\alpha_{n-1} , \alpha_{n}}^{[n] j_n } \; M_{\alpha_{n-1} ' , \alpha_{n} '}^{[n] j_n ' *} \; M_{\alpha_{n} , \alpha_{n+1}}^{[n+1] j_{n+1} } \; M_{\alpha_{n} ' , \alpha_{n+1} '}^{[n+1] j_{n+1} ' *}  \label{eq:twositeHamil}\\
	\braket{\psi | \psi} &= \sum_{j_n , j_{n+1} } \sum_{\substack{\alpha_{n-1} ' \\ \alpha_n ', \alpha_{n+1} '}} \sum_{\substack{\alpha_{n-1} , \alpha_n \\ \alpha_{n+1}}} \Psi_{\alpha_{n-1},\alpha_{n-1}'}^{A} \; M_{\alpha_{n-1} , \alpha_{n}}^{[n] j_n } \; M_{\alpha_{n-1} ' , \alpha_{n} '}^{[n] j_n ' *} \; M_{\alpha_{n} , \alpha_{n+1}}^{[n+1] j_{n+1} } \; M_{\alpha_{n} ' , \alpha_{n+1} '}^{[n+1] j_{n+1} ' *} \; \Psi_{\alpha_{n+1},\alpha_{n+1}'}^{B} \; , \label{eq:twositeOverlap}
\end{align}
where the Hamiltonian from equation \ref{eq:HPsi} has been re-ordered to accommodate for examining two sites, $n$ and $n+1$, at a time, and 
\begin{align}
\Psi_{\alpha_{n-1},\alpha_{n-1}'}^{A} &= \sum_{j_1 , \ldots , j_{n-1}} \left( M^{j_{n-1} \dag} \ldots M^{j_{1} \dag} M^{j_1} \ldots M^{j_{n-1}} \right) _{\alpha_{n-1} , \alpha_{n-1} '} \label{eq:psiA} \\
\Psi_{\alpha_{n+1},\alpha_{n+1}'}^{B} &= \sum_{j_{n+2} , \ldots , j_{N}} \left( M^{j_{n+2} } \ldots M^{j_{N} } M^{j_N \dag} \ldots M^{j_{n+2} \dag} \right) _{\alpha_{n+1} ', \alpha_{n+1} } \; .
\end{align}
Further simplifications can be made for mixed-canonical forms, if sites $1$ through $n-1$ are left-normalized, and sites $n+2$ through $N$ are right-normalized
\begin{equation}
	\Psi_{\alpha_{n-1},\alpha_{n-1}'}^{A} = \delta_{\alpha_{n-1},\alpha_{n-1}'} \qquad , \qquad \Psi_{\alpha_{n+1},\alpha_{n+1}'}^{B} = \delta_{\alpha_{n+1},\alpha_{n+1}'} \; .
\end{equation}
From here one has to find the extremum of equation \ref{eq:lagrange} with respect to $M_{\alpha_{n-1} ' , \alpha_{n} '}^{[n] j_n ' *} \; M_{\alpha_{n} ' , \alpha_{n+1} '}^{[n+1] j_{n+1} ' *}$. Looking at equation \ref{eq:twositeHamil} and \ref{eq:twositeOverlap} this may seem like a daunting task, however, it can actually be done fairly easily through the following sequence:

\subsection{Two-site update for iterative ground state search}
\begin{enumerate}
\item
\textbf{Merge:} Contract the two matrices $M^{[n]}$ and $M^{[n+1]}$ over the bond $\alpha_{n}$ creating a two-site tensor
\begin{equation}
\Theta_{\alpha_{n-1} , \alpha_{n+1}}^{j_n , j_{n+1}} = \sum_{\alpha_n} M_{\alpha_{n-1} , \alpha_{n}}^{[n] j_n } \;  M_{\alpha_{n} , \alpha_{n+1}}^{[n+1] j_{n+1} } 
\end{equation}

\item
\textbf{Solve eigenproblem:} The problem is simply an eigenvalue problem, which can be seen by reshaping
\begin{align}
	H_{( \alpha_{n-1}  j_n  j_{n+1}, \alpha_{n+1}),(\alpha_{n-1}'  j_n '  j_{n+1}', \alpha_{n+1}')} &= \nonumber \\
	= \; \sum_{\substack{\beta_{n-1} , \beta_n \\ \beta_{n+1}}} L_{\alpha_{n-1}, \beta_{n-1} , \alpha_{n-1} '}^{[n-1]} & \; W_{\beta_{n_1}, \beta_n}^{[n] j_n , j_n '} \; W_{\beta_{n}, \beta_{n+1}}^{[n+1] j_{n+1} , j_{n+1} '}\;  R_{\alpha_{n+1} ,\beta_{n+1} , \alpha_{n+1} '}^{[n+2]} \\
	v_{ \alpha_{n-1} j_n j_{n+1} \alpha_{n+1}} &= \; \Theta_{\alpha_{n-1} , \alpha_{n+1}}^{j_n , j_{n+1}}
\end{align}
such that
\begin{equation}
	H v - \lambda v = 0 \; .
	\label{eq:eigprob}
\end{equation}
Solving this for the lowest eigenvalue $\lambda_0$ yields $v_{ \alpha_{n-1} j_n j_{n+1} \alpha_{n+1}}^0$, which can be reshaped back to the now optimized $\tilde{\Theta}_{\alpha_{n-1} , \alpha_{n+1}}^{j_n , j_{n+1}}$.

\item
\textbf{Unmerge:} Reshape the updated $\tilde{\Theta}_{\alpha_{n-1} , \alpha_{n+1}}^{j_n , j_{n+1}}$ to a matrix and perform an SVD yielding
\begin{equation}
	\tilde{\Theta}_{(j_n \alpha_{n-1} ) ,(j_{n+1}  \alpha_{n+1} )} = \sum_{\alpha_n} U_{\alpha_{n-1} , \alpha_{n}}^{j_n} S_{\alpha_n , \alpha_n} (V^{\dag})_{\alpha_{n} , \alpha_{n+1}}^{j_{n+1}} \; .
\end{equation}
This causes the bond dimension to increase $D \rightarrow d D$, which must be truncated by keeping only the $D$ largest singular values of $S$. 

\item
\textbf{Update environments:} The last step depends on which direction one is iterating trough the chain. Here the left- and right-normalization of $U$ and $V^{\dag}$ is used to update the environments.\\
\textit{Going right}: Update the left environment
\begin{equation}
	\tilde{L}_{\alpha_{n}, \beta_{n} , \alpha_{n} '}^{[n]} = \sum_{\substack{ j_{n} , j_{n} ' \\ \alpha_{n-1} , \beta_{n-1} , \alpha_{n-1} '}} L_{\alpha_{n-1}, \beta_{n-1} , \alpha_{n-1} '}^{[n-1]} \; U_{\alpha_{n-1} , \alpha_{n}}^{j_n} \; W_{\beta_{n-1} , \beta_{n}}^{[n] j_n , j_n '} \; U_{\alpha_{n-1} ', \alpha_{n}'}^{j_n ' *} \; ,
\end{equation}
and build the matrix of the right site
\begin{equation}
	\tilde{M}_{\alpha_{n} , \alpha_{n+1}}^{[n+1] j_{n+1} } = \sum_{\alpha_n}  S_{\alpha_n , \alpha_n} (V^{\dag})_{\alpha_{n} , \alpha_{n+1}}^{j_{n+1}} \; .
\end{equation}\\ 
\textit{Going left}: Update the right environment
\begin{equation}
	\tilde{R}_{\alpha_{n}, \beta_{n} , \alpha_{n} '}^{[n+1]} = \sum_{\substack{ j_{n+1} , j_{n+1} ' \\ \alpha_{n+1} , \beta_{n+1} , \alpha_{n+1} '}} R_{\alpha_{n+1}, \beta_{n+1} , \alpha_{n+1} '}^{[n+2]} \; \left( V^{\dag} \right)_{\alpha_{n} , \alpha_{n+1}}^{j_{n+1}} \; W_{\beta_{n} , \beta_{n+1}}^{[n+1] j_{n+1} , j_{n+1} '} \; \left( V^{\dag} \right)_{\alpha_{n} , \alpha_{n+1}}^{j_{n+1} *} \; ,
\end{equation}
and build the matrix of the left site
\begin{equation}
	\tilde{M}_{\alpha_{n-1} , \alpha_{n}}^{[n] j_{n} } = \sum_{\alpha_n} U_{\alpha_{n-1} , \alpha_{n}}^{j_n} S_{\alpha_n , \alpha_n}  \; .
\end{equation} 
 
\end{enumerate}
This concludes the two-site update sequence, which can be seen illustrated in figure \ref{fig:twoSiteUpdate}. After performing the sequence, one can move \textit{one site} to either left or right, depending on which direction one is iterating.

\renewcommand{\thesubfigure}{\arabic{subfigure}}
\begin{figure}[h!]
	\centering
	\begin{subfigure}{\textwidth}
		\centering
		\caption{\textbf{Merge:}}
		\begin{tikzpicture}[inner sep=1mm]
	\def \imgcenter {5.25};
	\def \imgwidth {15};
	\node[minimum width=\imgwidth cm] (fake) at (\imgcenter ,0) {};


    \node[tensor] (M1) at (1 , 0) {};
    \node[tensor] (M2) at (2.5 , 0) {};
	\node[tensor, minimum width=2cm] (sig) at (9, 0) {};
	
	\node (j1) at (1 ,-1) {$j_n$};
	\node (j2) at (2.5 ,-1) {$j_{n+1}$};
	\node (a1) at (-0.2 ,0) {$\alpha_{n-1}$};
	\node (a2) at (3.7 ,0) {$\alpha_{n+1}$};    
    
	\draw[-] (M1) -- (j1);
	\draw[-] (M1) -- (a1);
	\draw[-] (M1) -- (M2);
	\draw[-] (M2) -- (j2);
	\draw[-] (M2) -- (a2);
	
	
	\node (node1) at (8.35, -1) {$j_n$};
	\node (node2) at (9.65, -1) {$j_{n+1}$};
	\node (as1) at (7 ,0) {$\alpha_{n-1}$};
	\node (as2) at (11 ,0) {$\alpha_{n+1}$};
	
	\draw[-] (node1) -- (node1 |-  sig.south);
	\draw[-] (node2) -- (node2 |-  sig.south);
	\draw[-] (sig) -- (as1);
	\draw[-] (sig) -- (as2);		
	
	\draw[->, line width=1mm] (4.5,0) -- (6,0);
	
	\node (Mlab1) at (1,0.8) {$M^{[n]}$};
	\node (Mlab2) at (2.5,0.8) {$M^{[n+1]}$};
	\node (Siglab) at (9,0.6) {$\Theta$};
	
	
\end{tikzpicture}
	\end{subfigure}\\[.8cm]
	
	\begin{subfigure}{\textwidth}
		\centering
		\caption{\textbf{Solve eigenprobem:}}
		\begin{tikzpicture}[inner sep=1mm]
	\def \imgcenter {5.8};
	\def \imgwidth {15};
	\node[minimum width=\imgwidth cm] (fake) at (\imgcenter ,0) {};	
	
	
	\node[tensor, minimum width=2.5cm] (theta) at (2, 0) {};
	\node[operator] (W1) at (1.1,-1.5) {};
	\node[operator] (W2) at (2.9,-1.5) {};	
	
	\node (dummy1) at (0.3,-3) {};
 	\node (dummy2) at (3.6,-3) {};
	\node[tensor] (L) at (-0.5 , -1.5) {};
	\node[tensor] (R) at (4.5 , -1.5) {};    
    
    \draw[-] (dummy1.west) .. controls (-0.5, -3) .. (L.south);
    \draw[-] (dummy2.east) .. controls (4.5, -3) .. (R.south);
    \draw[-] (theta.west) .. controls (-0.5, 0) .. (L.north);
    \draw[-] (theta.east) .. controls (4.5, 0) .. (R.north);
	
	\draw[-] (L) -- (W1);
	\draw[-] (W1) -- (W2);
	\draw[-] (W2) -- (R);
	\draw[-] (W1) -- (W1 |-  theta.south);
	\draw[-] (W2) -- (W2 |-  theta.south);
	
	\node (j1) at (1.1 , -2.5) {$j_{n}'$};
	\node (j2) at (2.9 , -2.5) {$j_{n+1}'$};
	\draw[-] (W1) -- (j1);
	\draw[-] (W2) -- (j2);
	
	\node (a1) at (0.5 , -3.3) {$\alpha_{n-1}'$};
	\node (a2) at (3.6 , -3.3) {$\alpha_{n+1}'$};
	\node (Llab) at (-1.2 , -1.5) {$L$};
	\node (Rlab) at (5 , -1.6) {$R$};	
	\node (Thetalab) at (2 , 0.6) {$\tilde{\Theta}$};
	\node (W1lab) at (0.6 , -0.9) {$W^{[n]}$};
	\node (W2lab) at (3.6 , -0.9) {$W^{[n+1]}$};
	
	
	\def \offsetx {7.5};
	\def \offsety {-1};
	\node[tensor, minimum width=2.5cm] (theta2) at (2+\offsetx, \offsety) {};
	\node (d1) at (1.1 +\offsetx, -1+ \offsety) {};
	\node (d2) at (2.9 +\offsetx, -1 + \offsety) {};
	\node (d3) at (2+\offsetx - 1.25 - 0.8, 0+\offsety) {};
	\node (d4) at (2+\offsetx + 1.25 + 0.8, 0+\offsety) {};
	
	\draw[-] (d1) -- (d1 |-  theta2.south);
	\draw[-] (d2) -- (d2 |-  theta2.south);
	\draw[-] (d3) -- (theta2);
	\draw[-] (d4) -- (theta2);
	
	\node (Thetalab2) at (2 +\offsetx, 0.6 + \offsety) {$\tilde{\Theta}$};
	
	
	\node (eq) at (5.8 , -1.5) {$=$};
	\node (lambda) at (7 , -1.5) {$\lambda_0 \;  \times$};	
\end{tikzpicture}
	\end{subfigure}\\[.8cm]

	\begin{subfigure}{\textwidth}
		\centering
		\caption{\textbf{Unmerge:}}
		\begin{tikzpicture}[inner sep=1mm]	
	\def \center {2};
	\def \width {2.5};
	\def \offsetx {8};
	\def \xspace {1};
	\def \imgcenter {\offsetx -2.75};
	\def \imgwidth {15};
	\node[minimum width=\imgwidth cm] (fake) at (\imgcenter ,0) {};
	
	
	\node[tensor, minimum width=\width cm] (theta) at (\center, 0) {};
	
	\node (d1) at (\center - \width/2 + 0.35,-1) {};
	\node (d2) at (\center + \width/2 - 0.35,-1) {};
	\node (d3) at (\center - \width/2 - 0.8, 0) {};
	\node (d4) at (\center + \width/2 + 0.8, 0) {};
	
	\draw[-] (d1) -- (d1 |-  theta.south);
	\draw[-] (d2) -- (d2 |-  theta.south);
	\draw[-] (d3) -- (theta);
	\draw[-] (d4) -- (theta);	
	
	\node (thetalab) at (\center , 0 + 0.6) {$\tilde{\Theta}$};
	
	
	\node[tensor] (U) at (\offsetx , 0) {};
	\node[matrix] (S) at (\offsetx + \xspace, 0) {};
    \node[tensor] (V) at (\offsetx + 2*\xspace , 0) {};
	
	
	\node (j1) at (\offsetx ,-1) {$j_n$};
	\node (j2) at (\offsetx + 2*\xspace ,-1) {$j_{n+1}$};
	\node (a1) at (\offsetx -1.2 ,0) {$\alpha_{n-1}$};
	\node (a2) at (\offsetx + 2*\xspace +1.2 ,0) {$\alpha_{n+1}$};    
    
	\node (Ulab) at (\offsetx , 0.6) {$U$};
	\node (Slab) at (\offsetx + \xspace , 0.6) {$S$};
	\node (Vlab) at (\offsetx + 2*\xspace , 0.6) {$V^{\dag}$};    
    
	\draw[-] (U) -- (j1);
	\draw[-] (U) -- (a1);
	\draw[-] (U) -- (S);
	\draw[-] (V) -- (S);
	\draw[-] (V) -- (j2);
	\draw[-] (V) -- (a2);
	
	\draw[->, line width=1mm] (\offsetx-3.5 ,0) -- (\offsetx-2,0);
	\node (SVD) at (\offsetx-2.9 ,0.4) {SVD};
\end{tikzpicture}
	\end{subfigure}\\[.8cm]

	\begin{subfigure}{\textwidth}
		\centering
		\caption{\textbf{Update environments:}}
		\begin{tikzpicture}[inner sep=1mm]
	\def \offsetx {6};
	\def \offsety {-5};
	\def \imgcenter {\offsetx -2.25};
	\def \imgwidth {15};
	\node[minimum width=\imgwidth cm] (fake) at (\imgcenter ,0) {};
	

	\Left{-0.5}{-1.5}{1.5};
	\node[tensor] (U1) at (1,0) {};	
	\node[tensor] (U2) at (1,-3) {};	
	\node[operator] (W) at (1,-1.5) {};

	\node (a1) at (2,0) {$\alpha_n$};
	\node (a2) at (2,-3) {$\alpha_n '$};
	\node (b) at (2,-1.5) {$\beta_n$};
	
	\draw[-] (U1) -- (W);
	\draw[-] (U2) -- (W);
	\draw[-] (U1) -- (a1);
	\draw[-] (U2) -- (a2);
	\draw[-] (W) -- (b);
	
	\node (L1) at (-1.4,-1.5) {$L^{[n-1]}$};
	\node (U1lab) at (1,0.6) {$U$};
	\node (U2lab) at (1,-3.6) {$U^{\dag}$};
	
	
	\draw[->, line width=1mm] (\offsetx-3 ,-1.5) -- (\offsetx-1.5,-1.5);
	
	
	\Left{\offsetx}{-1.5}{1.2};
	\node (L2) at (\offsetx -0.8,-1.5) {$\tilde{L}^{[n]}$};
	\node (a1) at (\offsetx+1.5,0) {$\alpha_n$};
	\node (a2) at (\offsetx+1.5,-3) {$\alpha_n '$};
	\node (b) at (\offsetx+1.5,-1.5) {$\beta_n$};
	
	%%----------------%%
	
	\Right{1}{-1.5 + \offsety}{1.5};
	\node[tensor] (U1) at (-0.5,0 + \offsety) {};	
	\node[tensor] (U2) at (-0.5,-3 + \offsety) {};	
	\node[operator] (W) at (-0.5,-1.5 + \offsety) {};

	\node (a1) at (-1.8,0 + \offsety) {$\alpha_{n}$};
	\node (a2) at (-1.8,-3 + \offsety) {$\alpha_{n} '$};
	\node (b) at (-1.8,-1.5 + \offsety) {$\beta_{n}$};
	
	\draw[-] (U1) -- (W);
	\draw[-] (U2) -- (W);
	\draw[-] (U1) -- (a1);
	\draw[-] (U2) -- (a2);
	\draw[-] (W) -- (b);
	
	\node (R1) at (2,-1.5 + \offsety) {$R^{[n+2]}$};
	\node (V1lab) at (-0.5,0.6 + \offsety) {$V^{\dag}$};
	\node (V2lab) at (-0.5,-3.6 + \offsety) {$V$};
	
	
	\draw[->, line width=1mm] (\offsetx-3 ,-1.5 + \offsety) -- (\offsetx-1.5,-1.5 + \offsety);
	
	
	\Right{\offsetx + 1.2}{-1.5 + \offsety}{1.2};
	\node (R2) at (\offsetx +2.2,-1.5 + \offsety) {$\tilde{R}^{[n+1]}$};
	\node (a1) at (\offsetx -0.5,0 + \offsety) {$\alpha_{n}$};
	\node (a2) at (\offsetx -0.5,-3 + \offsety) {$\alpha_{n} '$};
	\node (b) at (\offsetx -0.5,-1.5 + \offsety) {$\beta_{n}$};
\end{tikzpicture}
	\end{subfigure}
	
	
	\caption{\textit{A pictorial representation of the two-site update sequence for iterative ground state search. In step (4) the direction one is iterating determines whether to update the left or right environment.}}
	\label{fig:twoSiteUpdate}
\end{figure}

Some comments regarding this sequence are in order: The matrices of the eigenvalue problem has dimensions $( d^2 D^2 \times d^2 D^2)$, which is generally too much for exact diagonalization, however, since only the lowest eigenvalue is of interest, one can use an iterative eigensolver. Furthermore, if the MPS is not in the proper mixed-canonical form, the eigenvalue problem turns into a generalized eigenvalue problem, which can be numerically quite demanding. Thus, updating the left and right environments in step (4) is necessary, since it leads to great simplifications in step (2). Lastly, one could also consider just a single site when updating, however, this method is very prone to getting stuck \cite{WhiteSingleSite}. By updating two sites at once, one actually optimized the bond between them. Hence, after updating the two sites \textit{one must only iterate a single site} in order to optimize the next bond. Optimization is done with regards to the current configuration, whereby it depends on previous updates. To compensate for this one must sweep through the entire system multiple times, which leads to the following algorithm for iterative ground state search:

\subsection{Iterative grounds state search algorithm}
\begin{itemize}
\item[]
\textbf{Initialize:} Start from some right-normalized initial guess for $\ket{\psi}$. Using this calculate the tensor $R^{[i]}$ iteratively for all sites starting at the right.

\item[]
\textbf{Right sweep:} Starting from site $n = 1$ perform a \textit{two-site update}, where one updates the \textit{left environment} as described in the sequence. Afterwards $n$ is iterated by 1, $n \rightarrow n + 1$, and the process is repeated until $n = N-1$ is reached.

\item[]
\textbf{Left sweep:} Starting from site $n = N-1$ perform a \textit{two-site update}, where one updates the \textit{right environment} as described in the sequence. Afterwards $n$ is iterated by 1, $n \rightarrow n - 1$, and the process is repeated until $n = 1$ is reached.

\item[]
\textbf{Repeat sweeps:} Continue by alternating between left and right sweeps through the system until convergence is achieved.  
\end{itemize}