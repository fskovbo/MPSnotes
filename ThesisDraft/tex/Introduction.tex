\chapter{Introduction}

Ultracold atoms present an extremely powerful tool in quantum reasearch. At the scale of nanokelvins thermal effects no longer destroy the otherwise fragile quantum systems. Thus, quantum phenomena, otherwise only known from theory, can be studied directly. Atomic physics experiments with such ultracold, quantum degenerate gases offer some unique features: (i) a wide range of Hamiltonians can be mapped to the systems making experiments highly customizable. This can in part be attributed to (ii) the very high degree of control achievable through the manipulation of external fields. Through this, the cold atoms can be trapped and manipulated using magnetic and optical traps. Furthermore, the collisional properties of the atoms can be tuned through magnetic Feshbach resonances, and the properties of the gas can be probed through interactions between internal energy levels of the atoms and laser light \cite{JakschZoller, Bloch2012}.\\
In the regime of such cold temperatures many-body phenomena such as Bose-Einstein condensation takes place, where the ground state of a system gains a macroscopic populations. Ever since the first realisation of a Bose-Einstein condensate in 1995 \cite{WiemanCornell1995}, the special properties of this macroscopic quantum state have been used in a wide range of experiments. One method of utilizing Bose-Einstein condensates is to load them into an optical lattice, which is an array of potentials created through the atoms dipole interaction with laser beams. By utilizing the high controllability of these systems, one can perform quantum simulations of various systems, such as spin chains \cite{Simon2011}, Dirac cones \cite{Tarruell2012}, and artificial gauge fields \cite{Dalibard2011}. Furtermore, the properties of cold atoms in optical lattices is very favourable for experiments in quantum information, where quantum gates can be realised through controlled collisions \cite{Zoller1999} or Rydberg atoms \cite{Molmer2010}.\\

This note presents the basic theory of quantum many-body systems in optical lattices.
\newpage