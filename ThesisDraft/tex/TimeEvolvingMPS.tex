\chapter{Performing Time Evolution of Matrix Product States}
A central component of Optimal Control is time evolution. In order to compute any optimal control sequence, it is crucial to have a fast and accurate time evolution algorithm. However, also the construction of the time evolution operator must be taken into account, as exponentiating a tensor spanning the entire system, such as most Hamiltonians, is no easy task. In fact, since the time evolution operator is constantly altered when optimizing control parameters, the time spent exponentiating the Hamiltonian must be taken into account of the total runtime. Thus, both an efficient time evolution algorithm and an efficient operator exponentiation is needed when performing optimal control.

\section{The tDMRG Algorithm}
% Describe Trotter decomposition 
% Time evolve bonds rather than tensors - similar to DMRG
% Briefly describe tDMRG sequence using the already described DMRG algorithm
% Motivate modified algorithm: -exponentiation time -system specific
%   (can get better results building the algorithm around the system/
%    /optimication problem, rather than using a generic algorithm)
 
Several algorithms for time evolving matrix product states exist, however, they all origin from the same ideas proposed in \cite{Vidal2003,Vidal2004}. The most widely used of these algorithms is the tDMRG algorithm, which gets its name from its similarity with the ground state search algorithm described in Section \ref{sec:DMRG}. In fact, one only 



Consider the time evolution of a quantum state
\begin{equation}
	\ket{\psi (t)} = \hat{U}(t) \ket{\psi (t)} \; ,
\end{equation}
where $\hat{U}(t) = \e^{ - \im \hat{H} t }$ is the time evolution operator. Simply writing this operator as an MPO and applying it is not straightforward, however, if the Hamiltonian contains only nearest-neighbour interactions this can be achieved.
Assume the Hamiltonian is a sum of two-site operators of the form $\hat{H} = \sum_{n} \hat{h}^{[n , n+1]}$. One can decompose this into a sum over even and odd bonds
\begin{equation}
	\hat{H} = \hat{H}_{\mathrm{odd}} \; + \; \hat{H}_{\mathrm{even}} = \sum_{n \; \mathrm{odd}} \hat{h}^{[n , n+1]} \; + \; \sum_{n \; \mathrm{even}} \hat{h}^{[n , n+1]} \; .
\end{equation}  
From here one can discretize the time into small time slices $\delta t$ and perform a Trotter-Suzuki decomposition \cite{Suzuki}
\begin{equation}
	\e^{- \im \hat{H} \; \delta t} = \e^{- \im \hat{H}_{\mathrm{odd}} \; \delta t } \e^{- \im \hat{H}_{\mathrm{even}} \; \delta t} \; + \; \;  \mathrm{O}(\delta t^2) \; ,
\end{equation}
where the error is due to the noncommutativity of the bond Hamtiltonians $[ \hat{h}_{\mathrm{odd}}^{[n , n+1]} \; , \; \hat{h}_{\mathrm{even}}^{[n , n+1]} ] \neq 0$. Using this the time evolution operator can be written as the product
\begin{equation}
	\hat{U}(\delta t) \approx \left( \prod_{n \; \mathrm{odd}} \hat{U}^{[n,n+1]} (\delta t) \right) \left( \prod_{n \; \mathrm{even}} \hat{U}^{[n,n+1]} (\delta t) \right) \; ,
\end{equation}
where
\begin{equation}
	\hat{U}^{[n,n+1]} (\delta t) = \e^{- \im \hat{h}^{[n , n+1]} \; \delta t } \; .
\end{equation}
Each of these bond evolutions can be written in the form
\begin{equation}
	 U^{(j_n j_{n+1}), (j_n ' j_{n+1} ')} =  \bra{j_n j_{n+1}} \e^{- \im \hat{h}^{[n , n+1]} \; \delta t } \ket{j_n ' j_{n+1} '} \; ,
\end{equation}
whereby it can be represented as an MPO. Thus, one has one MPO performing an infinitesimal time step on the odd bonds and another MPO doing the same for even bonds, which can be seen in figure \ref{fig:oddevenops}.
\begin{figure}[h!]
	\centering
	\begin{tikzpicture}[inner sep=1mm]
	\def \reldist {1.5};
	\def \numb {6};
	\def \wid {2}

	\foreach \i in  {1,...,\numb} {
		\node[tensor] (t\i) at (\i * \reldist, 0) {};
		\draw[-] (t\i) -- (\i * \reldist , -2.8);	
	};
	
	\foreach \i in {1,...,5} {
        \pgfmathtruncatemacro{\iplusone}{\i + 1};
        \draw[-] (t\i) -- (t\iplusone);
	};
	
	\foreach \i in {1,3,5} {
        \node[twositeop, minimum width= \wid cm] (op\i) at (\reldist*\i + \wid/2 -0.25,-1) {$\hat{\mathcal{U}}^{\mathrm{even}} (\Delta t)$};
	};
	
	\foreach \i in {2,4,6} {
        \node[twositeop, minimum width= \wid cm] (op\i) at (\reldist*\i + \wid/2 -0.25,-2) {$\hat{\mathcal{U}}^{\mathrm{odd}} (\Delta t)$};
	};	
	
	\node (dot1) at (0,0) {$\dots$};
	\node (dot2) at (\numb * \reldist + \reldist,0) {$\dots$};
	\draw[-] (t1) -- (dot1);
	\draw[-] (t\numb) -- (dot2);
	
	\draw[decoration={calligraphic brace,amplitude=10pt}, decorate, line width=1.25pt, xshift=-4pt, yshift=0pt]
(0, -2.5) -- (0,-0.5) node [black,midway,xshift=-0.6cm] 
{\large $\Delta t$};
	
\end{tikzpicture}
	\caption{\textit{Approximation of each time step $\delta t$ using a Trotter-Suzuki decomposition, such that the time evolution operator is expressed as a product of unitary two-site operators.}}
	\label{fig:oddevenops}
\end{figure}

Consider an MPS in the mixed-canonical form
\begin{equation}
	\ket{\psi} = \sum_{\boldsymbol{j}} A^{j_1} \ldots A^{j_{n-1}} \; M^{j_n} S M^{j_{n+1}} \; B^{j_{n+2}} \ldots B^{j_N} \ket{\boldsymbol{j}} \;
\end{equation}
where the matrices $M$ can have any form. Staring from this state the tDMGR update sequence looks as follows:

\subsection{Infinitesimal time-step update for tDMRG}
\begin{enumerate}
\item
\textbf{Merge:} Contract the two matrices $M^{[n]}$ and $M^{[n+1]}$ over the bond $\alpha_{n}$ creating a two-site tensor
\begin{equation}
	\Phi_{\alpha_{n-1} , \alpha_{n+1}}^{j_n , j_{n+1}} = \sum_{\alpha_n} M_{\alpha_{n-1} , \alpha_{n}}^{[n] j_n } \; S_{\alpha_{n} , \alpha_{n}} \;  M_{\alpha_{n} , \alpha_{n+1}}^{[n+1] j_{n+1} } 
\end{equation}


\item
\textbf{Apply unitary:} The two-site time evolution operator, $U^{j_n  j_{n+1} , j_n '  j_{n+1}'}$, can now easily be applied to the two-site tensor
\begin{equation}
	\tilde{\Phi}_{\alpha_{n-1} , \alpha_{n+1}}^{j_n , j_{n+1} } = \sum_{j_n ', j_{n+1}'} U^{j_n  j_{n+1} , j_n '  j_{n+1}'} \; \Phi_{\alpha_{n-1} , \alpha_{n+1}}^{j_n ', j_{n+1} ' } \; .
\end{equation}


\item
\textbf{Unmerge:} Reshape the updated $\tilde{\Phi}_{\alpha_{n-1} , \alpha_{n+1}}^{j_n ', j_{n+1} '}$ to a matrix and perform an SVD yielding
\begin{equation}
	\tilde{\Phi}_{(j_n  \alpha_{n-1} ) ,(j_{n+1}  \alpha_{n+1})} = \sum_{\alpha_n} U_{\alpha_{n-1} , \alpha_{n}}^{j_n } S_{\alpha_n , \alpha_n} (V^{\dag})_{\alpha_{n} , \alpha_{n+1}}^{j_{n+1} } \; .
\end{equation}
This causes the bond dimension to increase $D \rightarrow d D$, which must be truncated by keeping only the $D$ largest singular values of $S$. 


\item
\textbf{Progress:}  Next, $\e^{- \im \hat{h}^{[n , n+1]} \; \delta t }$ must be applied to the next pair of sites, thus one has to shift by two sites to continue.\\
First, keep $U_{\alpha_{n-1} , \alpha_{n}}^{j_n } = A_{\alpha_{n-1} , \alpha_{n}}^{[n] j_n }$ and form the shifted $\Phi$ as
\begin{equation}
	\Phi_{\alpha_{n} , \alpha_{n+2}}^{j_{n+1} , j_{n+2}} = \sum_{\alpha_{n+1}}  S_{\alpha_n , \alpha_n} (V^{\dag})_{\alpha_{n} , \alpha_{n+1}}^{j_{n+1} } B_{\alpha_{n+1} , \alpha_{n+1}}^{j_{n+2}} \; ,
\end{equation}
This is just a shift by one site, however, by reshaping and performing a second SVD one can form the $\Phi$ shifted by two sites
\begin{equation}
	\Phi_{\alpha_{n+1} , \alpha_{n+3}}^{j_{n+2} , j_{n+3}} = \sum_{\alpha_{n+2}}  S_{\alpha_{n+1} , \alpha_{n+1}} (V^{\dag})_{\alpha_{n+1} , \alpha_{n+2}}^{j_{n+2} } B_{\alpha_{n+2} , \alpha_{n+3}}^{j_{n+3}} \; ,
\end{equation}
where $U_{\alpha_{n} , \alpha_{n+1}}^{j_{n+1} } = A_{\alpha_{n} , \alpha_{n+1}}^{[n+1] j_{n+1} }$, thus retaining the canonical form.\\
After each SVD the matrices are truncated down to dimension $D$, however, the second SVD causes no loss of information, as the Schmidt rank of $S$ will be at most $D$. 
\end{enumerate}
From here one can apply the unitary $U^{[n+2 , n+3]}$ to the new two-site tensor and proceed with the next infinitesimal time-evolution step all while remaining in the canonical form \cite{Schollwock}. This sequence is illustrated in figure \ref{fig:tDMRG}.

\section{Modified Time Evolution Algorithm for Optimal Control}
% Split of H into H_J and H_U and why (direct contruction of exp(H_U) )
% Sequence/algortihm
% graphs from testing

\begin{figure}[h!]
	\centering
	\begin{tikzpicture}[inner sep=1mm]
\def \reldist {2};
\def \numb {4};
\def \wid {3.2};
\def \size {1.0};
\def \hi {0.8};
\def \vert {1.0};
\def \rad {0.4};

	
\foreach \i in  {1,...,\numb} {
	\node[tensor,minimum width= \size cm,minimum height= \size cm, rounded corners = 0.2cm] (A\i)
	at (\i * \reldist, 0) {$A^{[ \i ]}$};
	\draw[-] (A\i) -- (\i * \reldist , -\numb*\vert -1.8);	
};

\foreach \i in {1,...,3} {
    \pgfmathtruncatemacro{\iplusone}{\i + 1};
    \draw[-] (A\i) -- (A\iplusone);
};

\foreach \i in {1,...,\numb} {
	\pgfmathtruncatemacro{\j}{\i + 1};
    \node[twositeop, minimum width= \wid cm,minimum height= \hi cm,rounded corners = \rad cm] (fop\i)
    at (\reldist*\i + \reldist/2, -\vert*\i -0.1)
    {\small $\hat{\mathcal{U}}_{J}^{[ \i , \j ]} (\delta t /2)$};
};
	
	
\foreach \i in  {1,...,\numb} {
	\node[operator,minimum width= \hi cm,minimum height= \hi cm] (op\i)
	at (\i * \reldist, -\vert*\i -0.1 - \vert)
	{\scriptsize $\hat{\mathcal{U}}_{U}^{[ \i ]}$};	
};


\foreach \i in  {1,...,\numb} {
	\node (dots\i) at (\i * \reldist, -\numb*\vert -2) {\normalsize $\vdots$};
	\draw[-] (dots\i) -- (\i * \reldist , -2*\numb*\vert -2.7);	
};

\foreach \i in {1,...,\numb} {
	\pgfmathtruncatemacro{\j}{\i + 1};
    \node[twositeop, minimum width= \wid cm,minimum height= \hi cm,rounded corners = \rad cm] (sop\i)
    at (\reldist*\i + \reldist/2, \vert*\i -2*\numb*\vert - 3)
    {\small $\hat{\mathcal{U}}_{J}^{[ \i , \j ]} (\delta t /2)$};
};



\draw[decoration={calligraphic brace,amplitude=10pt}, decorate, line width=1.25pt, xshift=-4pt, yshift=0pt]
(0, -2*\numb*\vert -2.5) -- (0,-0.8) node [black,midway,xshift=-0.6cm] 
{\large $\delta t$};


	\node (dot2) at (\numb * \reldist + 2.4,0) {$\dots$};

	\draw[-] (A\numb) -- (dot2);
	
	
\end{tikzpicture}
	\caption{\textit{Tensor diagram depicting a single time step of the modified TEBD algorithm. The time evolution operator has been subjected to a Suzuki-Trotter expansion as detailed in eq. \eqref{eq:SuzukiTrotter}. The tensors of the upper part of the network are contracted with the MPS while sweeping from left to right, whereas the lower part is applied with a right to left sweep.}}
	\label{fig:ModifiedTEBD}
\end{figure}


\begin{figure}
\centering % <-- add this
\begin{subfigure}[b]{0.35\textwidth}
	\caption{}  
  	\begin{tikzpicture}[inner sep=1mm]
\def \reldist {2};
\def \numb {4};
\def \wid {3.2};
\def \size {1.0};
\def \rad {0.5};



\node[tensorc,minimum width= \size cm,minimum height= \size cm, rounded corners = 0.2cm] (A1) at (1 * \reldist, 0) {$A^{[1]}$};

\node[tensorr,minimum width= \size cm,minimum height= \size cm, rounded corners = 0.2cm] (A2) at (2 * \reldist, 0) {$A^{[2]}$};

\node[tensorr,minimum width= \size cm,minimum height= \size cm, rounded corners = 0.2cm] (A3) at (3 * \reldist, 0) {$A^{[3]}$};

\node[tensorr,minimum width= \size cm,minimum height= \size cm, rounded corners = 0.2cm] (A4) at (4 * \reldist, 0) {$A^{[4]}$};

\foreach \i in  {1,...,\numb} {
	\draw[-] (A\i) -- (\i * \reldist , -8.5);	
};

\foreach \i in {1,...,3} {
    \pgfmathtruncatemacro{\iplusone}{\i + 1};
    \draw[-] (A\i) -- (A\iplusone);
};

\node[twositeop, minimum width= \wid cm,minimum height= \size cm,rounded corners = \rad cm] (fop1)
    at (\reldist*1 + \reldist/2,-1.5)
    {\small $\hat{\mathcal{U}}_{J}^{[1,2]}(\delta t /2)$};
    
\node[twositeop, minimum width= \wid cm,minimum height= \size cm,rounded corners = \rad cm] (fop3)
    at (\reldist*3 + \reldist/2,-1.5)
    {\small $\hat{\mathcal{U}}_{J}^{[3,4]}(\delta t /2)$};

\node[twositeop, minimum width= \wid cm,minimum height= \size cm,rounded corners = \rad cm] (fop2)
    at (\reldist*2 + \reldist/2,-3)
    {$\hat{\mathcal{U}}_{J}^{[2,3]}(\delta t /2)$};
	
	
\foreach \i in  {1,...,\numb} {
	\node[operator,minimum width= \size cm,minimum height= \size cm] (op\i)
	at (\i * \reldist, -4.5)
	{\scriptsize $\hat{\mathcal{U}}_{U}(\delta t)$};	
};

\node[twositeop, minimum width= \wid cm,minimum height= \size cm,rounded corners = \rad cm] (sop1)
    at (\reldist*1 + \reldist/2,-6)
    {\small $\hat{\mathcal{U}}_{J}^{[1,2]}(\delta t /2)$};
    
\node[twositeop, minimum width= \wid cm,minimum height= \size cm,rounded corners = \rad cm] (sop3)
    at (\reldist*3 + \reldist/2,-6)
    {\small $\hat{\mathcal{U}}_{J}^{[3,4]}(\delta t /2)$};

\node[twositeop, minimum width= \wid cm,minimum height= \size cm,rounded corners = \rad cm] (sop2)
    at (\reldist*2 + \reldist/2,-7.5)
    {$\hat{\mathcal{U}}_{J}^{[2,3]}(\delta t /2)$};



	\node (dot2) at (\numb * \reldist + 2.4,0) {$\dots$};
	\draw[-] (A\numb) -- (dot2);
	
	
\end{tikzpicture}
\end{subfigure}
\begin{subfigure}[b]{0.35\textwidth}
	\caption{}    
  	\begin{tikzpicture}[inner sep=1mm]
\def \reldist {2};
\def \numb {4};
\def \wid {3.2};
\def \size {1.0};
\def \rad {0.5};



\node[tensorc,minimum width= \size cm,minimum height= \size cm, rounded corners = 0.2cm] (A1) at (1 * \reldist, 0) {$A^{[1]}$};

\node[tensorr,minimum width= \size cm,minimum height= \size cm, rounded corners = 0.2cm] (A2) at (2 * \reldist, 0) {$A^{[2]}$};

\node[tensorr,minimum width= \size cm,minimum height= \size cm, rounded corners = 0.2cm] (A3) at (3 * \reldist, 0) {$A^{[3]}$};

\node[tensorr,minimum width= \size cm,minimum height= \size cm, rounded corners = 0.2cm] (A4) at (4 * \reldist, 0) {$A^{[4]}$};

\node[tensor,minimum width= \wid cm,minimum height= \size cm, rounded corners = 0.2cm] (AA) at (\reldist*1 + \reldist/2, 0) {$\Theta$};

\foreach \i in  {1,...,\numb} {
	\draw[-] (A\i) -- (\i * \reldist , -8.5);	
};

\foreach \i in {1,...,3} {
    \pgfmathtruncatemacro{\iplusone}{\i + 1};
    \draw[-] (A\i) -- (A\iplusone);
};

\node[twositeop, minimum width= \wid cm,minimum height= \size cm,rounded corners = \rad cm] (fop1)
    at (\reldist*1 + \reldist/2,-1.5)
    {\small $\hat{\mathcal{U}}_{J}^{[1,2]}(\delta t /2)$};
    
\node[twositeop, minimum width= \wid cm,minimum height= \size cm,rounded corners = \rad cm] (fop3)
    at (\reldist*3 + \reldist/2,-1.5)
    {\small $\hat{\mathcal{U}}_{J}^{[3,4]}(\delta t /2)$};

\node[twositeop, minimum width= \wid cm,minimum height= \size cm,rounded corners = \rad cm] (fop2)
    at (\reldist*2 + \reldist/2,-3)
    {$\hat{\mathcal{U}}_{J}^{[2,3]}(\delta t /2)$};
	
	
\foreach \i in  {1,...,\numb} {
	\node[operator,minimum width= \size cm,minimum height= \size cm] (op\i)
	at (\i * \reldist, -4.5)
	{\scriptsize $\hat{\mathcal{U}}_{U}(\delta t)$};	
};

\node[twositeop, minimum width= \wid cm,minimum height= \size cm,rounded corners = \rad cm] (sop1)
    at (\reldist*1 + \reldist/2,-6)
    {\small $\hat{\mathcal{U}}_{J}^{[1,2]}(\delta t /2)$};
    
\node[twositeop, minimum width= \wid cm,minimum height= \size cm,rounded corners = \rad cm] (sop3)
    at (\reldist*3 + \reldist/2,-6)
    {\small $\hat{\mathcal{U}}_{J}^{[3,4]}(\delta t /2)$};

\node[twositeop, minimum width= \wid cm,minimum height= \size cm,rounded corners = \rad cm] (sop2)
    at (\reldist*2 + \reldist/2,-7.5)
    {$\hat{\mathcal{U}}_{J}^{[2,3]}(\delta t /2)$};



	\node (dot2) at (\numb * \reldist + 2.4,0) {$\dots$};
	\draw[-] (A\numb) -- (dot2);
	
	
\end{tikzpicture}
\end{subfigure}
\\ % <-- add this
\vspace{10mm}
\begin{subfigure}[b]{0.35\textwidth}
	\caption{}    	
  	\begin{tikzpicture}[inner sep=1mm]
\def \reldist {2};
\def \numb {2};
\def \wid {3.2};
\def \size {1.0};
\def \rad {0.5};



\node[tensor,minimum width= \size cm,minimum height= \size cm, rounded corners = 0.2cm] (A1) at (1 * \reldist, 0) {$A^{[1]}$};

\node[tensor,minimum width= \size cm,minimum height= \size cm, rounded corners = 0.2cm] (A2) at (2 * \reldist, 0) {$A^{[2]}$};


\foreach \i in  {1,...,\numb} {
	\draw[-] (A\i) -- (\i * \reldist ,-2.5);	
};
   
\draw[-] (A1) -- (A2);    

\node[twositeop, minimum width= \wid cm,minimum height= \size cm,rounded corners = \rad cm] (fop2)
    at (\reldist*2 + \reldist/2, -1.5 )
    {$\hat{\mathcal{U}}_{J}^{[2,3]}(\delta t /2)$};
	
\node[operator,minimum width= \size cm,minimum height= \size cm] (op1)
	at ( \reldist, -1.5)
	{\scriptsize $\hat{\mathcal{U}}_{U}(\delta t)$};	

\draw[-,line width=0.8mm] (A1) -- (op1);

\node (dot2) at (\numb * \reldist + 1.4,0) {$\dots$};
\draw[-] (A\numb) -- (dot2);
	
	
\end{tikzpicture}
\end{subfigure}
\begin{subfigure}[b]{0.35\textwidth}
	\caption{}  
  	\begin{tikzpicture}[inner sep=1mm]
\def \reldist {1.5};
\def \numb {4};
\def \wid {2.3};
\def \size {1.0};
\def \hi {0.8};
\def \rad {0.4};
\def \vert {1.35};


\foreach \i in  {1,...,\numb} {
	\node[tensorr,minimum width= \size cm,minimum height= \size cm, rounded corners = 0.2cm] (A\i)
	at (\i * \reldist, 0) {$M^{[ \i ]}$};
	\draw[-] (A\i) -- (\i * \reldist , -2.5*\vert);	
};

\foreach \i in {1,...,3} {
    \pgfmathtruncatemacro{\iplusone}{\i + 1};
    \draw[-] (A\i) -- (A\iplusone);
};

\node[tensorl,minimum width= \size cm,minimum height= \size cm, rounded corners = 0.2cm] (L1) at (1 * \reldist, 0) {$M^{[1]}$};

\node[tensorl,minimum width= \size cm,minimum height= \size cm, rounded corners = 0.2cm] (L2) at (2 * \reldist, 0) {$M^{[2]}$};

\node[tensorc,minimum width= \size cm,minimum height= \size cm, rounded corners = 0.2cm] (C) at (3 * \reldist, 0) {$M^{[3]}$};


\foreach \i in  {1,...,\numb} {
	\draw[-] (A\i) -- (\i * \reldist , -2.6 *\vert);	
};

\draw[-,line width=0.8mm] (A3) -- (A4);

\foreach \i in  {3,...,\numb} {
	\node[operator,minimum width= \hi cm,minimum height= \hi cm] (op\i)
	at (\i * \reldist, -\vert)
	{\scriptsize $\hat{\mathcal{U}}_{U}^{[ \i ]}$};	
};

\foreach \i in {3} {
	\pgfmathtruncatemacro{\j}{\i + 1};
    \node[twositeop, minimum width= \wid cm,minimum height= \hi cm,rounded corners = \rad cm] (top\i)
    at (\reldist*\i + \reldist/2, -2*\vert)
    {\small $\hat{\mathcal{U}}_{J}^{[ \i , \j ]} $};
};


\node (dot2) at (\numb * \reldist + 1.4,0) {$\dots$};
\draw[-] (A\numb) -- (dot2);
	
	
\end{tikzpicture}
\end{subfigure}
\caption{\textit{Sequence of contractions for modified TEBD algorithm during left to right sweep. Step \textbf{(i)}: MPS centred on site 1, tensors $A^{[1]}$ and $A^{[2]}$ are contracted. Step \textbf{(ii)}: Two-site tensor is contracted with two-site operator, followed by a splitting of the resulting tensor through an SVD in step \textbf{(iii)}. Lastly, in step \textbf{(iv)}, the center (and thereby the normalisation) is pushed to the next site leaving $A^{[1]}$ left-normalised.}}
\label{fig:TEBDContraction}
\end{figure}