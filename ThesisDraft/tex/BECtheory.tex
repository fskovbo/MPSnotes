\chapter{Bose-Einstein Condensates}

Bosons are particles of integer spin, whose statistics obey those of a Bose-Einstein distribution
\begin{equation}
	n_i = \frac{g_i}{\exp \left( \left( \varepsilon_i -\mu \right) / k_B T \right) - 1} \; , \label{eq:BHdistribution}
\end{equation} 
where $i$ denotes the state, $n_i$ is the population of the state, $g_i$ is its degeneracy, $\varepsilon_i$ is its energy, $\mu$ is the chemical potential, $k_B$ is the Boltzmann constant, and $T$ is the temperature. One important feature of bosons is that, unlike fermions, multiple particles can occupy the same quantum state. At higher temperatures the energy spectrum is practically continues, whereby this property has little effect, as two particles occupying the same single-particle state is highly unlikely. However, at low temperatures the energy spectrum systems often become increasingly discrete, hence the statistics of the particles becomes very important. As evident from equation \eqref{eq:BHdistribution}, the population of the ground state diverges as $T \to 0$. However, even below the finite temperature, $T_c$, one will observe a macroscopic population of the ground state. $T_c$ is called the critical temperature, and marks the point where multiple particles will start forming a Bose-Einstein Condensate (BEC). \cite{pethick2002bose}

\section{Non-Interacting Particles}
In the case of non-interacting particles and zero temperature, all particles of a Bose gas can be described by identical single-particles wavefunctions $\phi (\boldsymbol{r}_i)$. Hence, the many-body wavefunction is simply given by the product 
\begin{equation}
	\Psi (\boldsymbol{r}_1 , \ldots , \boldsymbol{r}_N) = \prod_{i}^{N} \phi (\boldsymbol{r}_i) \; .
\end{equation}
Such a product state can be described by a single macroscopic wavefunction
\begin{equation}
	\psi (\boldsymbol{r}) = \sqrt{N} \varphi (\boldsymbol{r}) \; , \label{eq:psi_NIBEC}
\end{equation}
where $\phi (\boldsymbol{r})$ is the wave function of the single-particle state, in which the bosons condensate into \cite{PenroseOnsager}.

\subsection{Second-Quantization}
When describing Bose-Einstein condensates it is very convenient to work in second quantization, which describes the number of particle in each state rather than the state of each particle.\\
First, consider a basis of single particle states $\{ \ket{n} \}$, namely a Fock basis. In this space particles are created or annihilated through their respective operators
\begin{equation}
	\hat{a}_{\mu}^{\dag} \ket{0_\mu} = \ket{1_\mu} \; .
\end{equation}
For bosons the creation and annihilation operators fulfill the commutation relations
\begin{equation}
[\hat{a}_\nu,\hat{a}_\mu]=[\hat{a}_\nu^\dagger,\hat{a}_\mu^\dagger]=0 \quad , \quad [\hat{a}_\nu,\hat{a}_\mu^\dagger]=\delta_{\nu,\mu} \; , 
\end{equation}
with the number operator given as
\begin{equation}
	\hat{n}_{\mu} = \hat{a}_{\mu}^{\dag} \hat{a}_{\mu} \; .
\end{equation}
In second quantization many-body states are described by the occupation of the individual Fock states. Thus, a creation operator will raise the number of particles in its corresponding state by one, while the annihilation operator will lower it:
\begin{align}
\hat{a}^\dagger \ket{N_0,N_1, \ldots , N_{\nu},\ldots}&= \sqrt{N_\nu+1}\ket{N_0,N_1, \ldots , N_{\nu}+1,\ldots} \\
\hat{a} \ket{N_0,N_1, \ldots , N_{\nu},\ldots}&= \sqrt{N_\nu}\ket{N_0,N_1, \ldots , N_{\nu}-1,\ldots} .
\end{align}
Likewise, the number operator $\hat{n}_{\mu}$ will count the number of particles in its corresponding state.\\
These operators can be combined with an orthonormal basis of spatial wavefunctions $\{ \phi_k \}$ in order to create field operators
\begin{equation}
	\hat{\psi}(\boldsymbol{r}) = \; \sum_{k} \phi_k \hat{a}_{k} \quad , \quad \hat{\psi}^{\dag}(\boldsymbol{r}) = \; \sum_{k} \phi_{k}^{*} \hat{a}_{k}^{\dag} \; ,
\end{equation}
where $\hat{\psi}(\boldsymbol{r})$ will create a particle at location $\boldsymbol{r}$. For bosons the field operators fulfil the commutation relations \cite{bruus}
\begin{equation}
	\left[ \hat{\psi}(\boldsymbol{r}) \; , \; \hat{\psi}^{\dag}(\boldsymbol{r'}) \right] = \; \delta(\boldsymbol{r} - \boldsymbol{r}') \quad , \quad
	\left[ \hat{\psi}(\boldsymbol{r}) \; , \; \hat{\psi}(\boldsymbol{r'}) \right] = \; 0 \; .
\end{equation}
Now, consider a gas of non-interacting bosons described by equation \eqref{eq:psi_NIBEC}, where $\epsilon_k$ is the energy of the $k$'th single-particle state. Due to the completeness of the basis of single-particle wavefunctions, $\{ \phi_k \}$, the creation operator can be expressed as
\begin{equation}
	\hat{a}_k = \int \mathrm{d^3} r \;  \phi_{k}^*(\boldsymbol{r}) \hat{\psi}(\boldsymbol{r}) \; .
\end{equation}
Using this, the Hamiltonian can be written as
\begin{align}
	\hat{H}^{(0)} =& \; \sum_{k} \epsilon_k \hat{a}_{k}^{\dag} \hat{a}_{k} \nonumber \\
		=& \;  \sum_{k} \int \mathrm{d^3}r_1 \mathrm{d^3}r_2 \; \epsilon_k \phi_k (\boldsymbol{r_1}) \phi_{k}^* (\boldsymbol{r_2})\; \; \hat{\psi}^{\dag} (\boldsymbol{r_1}) \hat{\psi} (\boldsymbol{r_2}) \nonumber \\
		=& \; \int \mathrm{d^3}r  \; \hat{\psi}^{\dag}(\boldsymbol{r}) \left( - \frac{\hbar^2}{2 m} \nabla^2 + U(\boldsymbol{r})\right) \hat{\psi}(\boldsymbol{r})
		\label{hamil2nd}
\end{align}

\section{Weakly Interacting Particles}
A characteristic of a BEC is its low temperature and density. Thus, is it a valid approximation to only consider two-particle interactions
\begin{equation}
	\hat{H}^{(2)} = \frac{1}{2} \sum_{i \neq j} V(\boldsymbol{r_i} - \boldsymbol{r_j}) \; .
\end{equation}
At low energies all interactions can be considered of s-wave nature, because
all waves of higher angular momentum are reflected by the centrifugal barrier. Furthermore, for cold gases the thermal de Broglie wavelength is much larger than the effective extension of the interaction potential. Therefore, the actual shape of the scattering potential is irrelevant, hence one can replace it with a pseudo-potential
\begin{equation}
	V(\boldsymbol{r} - \boldsymbol{r'}) = g \; \delta(\boldsymbol{r} - \boldsymbol{r'}) = \frac{4 \pi \hbar^2 a}{m} \delta(\boldsymbol{r} - \boldsymbol{r'}) \; ,
\end{equation}
which results in the same scattering phase as the real, more complicated scattering potential. Thus, the interaction of cold atoms is fully determined by the scattering length, $a$ \cite{greiner}.\\
Introducing the density operator
\begin{equation}
	\hat{\rho}(\boldsymbol{r}) = \hat{\psi}^{\dag}(\boldsymbol{r}) \hat{\psi}(\boldsymbol{r}) \; ,
\end{equation}
allows writing $\hat{H}^{(2)}$ in second quantization
\begin{align}
	\hat{H}^{(2)} &= \frac{1}{2} \int \mathrm{d^3}r_1 \mathrm{d^3}r_2 V(\boldsymbol{r_1} - \boldsymbol{r_2}) \hat{\psi}^{\dag}(\boldsymbol{r_1}) \hat{\psi}(\boldsymbol{r_1}) \left( \hat{\psi}^{\dag}(\boldsymbol{r_2}) \hat{\psi}(\boldsymbol{r_2}) - \delta(\boldsymbol{r_1} - \boldsymbol{r_2}) \right) \\
	&= \frac{1}{2} \int \mathrm{d^3}r_1 \mathrm{d^3}r_2  \hat{\psi}^{\dag}(\boldsymbol{r_1}) \hat{\psi}^{\dag}(\boldsymbol{r_2}) V(\boldsymbol{r_1} - \boldsymbol{r_2}) \hat{\psi}(\boldsymbol{r_1}) \hat{\psi}(\boldsymbol{r_2}) \; .
\end{align}
Combining this with the basic Hamiltonian of equation \eqref{hamil2nd}, gives the full Hamiltonian in second quantization
\begin{align}
	\hat{H} &= \hat{H}^{(0)} + \hat{H}^{(2)} \\
	& = \int \mathrm{d^3}r \ \hat{\psi}^{\dag}(\boldsymbol{r}) \left( - \frac{\hbar^2}{2 m} \nabla^2 + U(\boldsymbol{r})\right) \hat{\psi}(\boldsymbol{r}) + \frac{1}{2} \int \mathrm{d^3}r_1 \mathrm{d^3}r_2  \ \hat{\psi}^{\dag}(\boldsymbol{r_1}) \hat{\psi}^{\dag}(\boldsymbol{r_2}) V(\boldsymbol{r_1} - \boldsymbol{r_2}) \hat{\psi}(\boldsymbol{r_1}) \hat{\psi}(\boldsymbol{r_2})
	\label{hamilint}
\end{align}
Using this Hamiltonian to try and solve the Heisenberg equations of motion for the field operators leads to
\begin{equation}
	i \hbar \frac{\partial }{\partial t} \hat{\psi}(\boldsymbol{r}) = \left[ \hat{\psi}(\boldsymbol{r}) \; , \; \hat{H}  \right] = \left( - \frac{\hbar^2}{2 m} \nabla^2 + U(\boldsymbol{r}) + g \hat{\psi}^{\dag}(\boldsymbol{r}) \hat{\psi}(\boldsymbol{r}) \right) \hat{\psi}(\boldsymbol{r}) \; ,
\end{equation}
which is not solvable in general. However, in the scenario of a BEC the scattering length $a$ is much less than the mean inter-particle distance, such that $n a^3 \ll 1$, where $n$ is the density of the gas. In this regime the mean-field approximation is viable
\begin{equation}
	\hat{\psi}(\boldsymbol{r}) = \psi(\boldsymbol{r}) + \delta \hat{\psi}(\boldsymbol{r}) \; ,
\end{equation}
where $\psi(\boldsymbol{r})$ is the mean-field given by equation \eqref{eq:psi_NIBEC}, and $\delta \hat{\psi}(\boldsymbol{r})$ is fluctuations from the mean. If $\braket{\delta \hat{\psi}(\boldsymbol{r})} = 0$, the fluctuations can be neglected, leading to the Gross-Pitaevskii equation \cite{Gross1961,Pitaevskii}
\begin{equation}
	i \hbar \frac{\partial }{\partial t} \hat{\psi}(\boldsymbol{r}) = \left( - \frac{\hbar^2}{2 m} \nabla^2 + U(\boldsymbol{r}) + g |\psi(\boldsymbol{r})|^2 \right) \psi(\boldsymbol{r}) \; .
\end{equation}
The Gross-Pitaevskii equation is very similar to the Schrödinger equation with exception of the non-linear term $g |\psi(\boldsymbol{r})|^2$, which can make the equation hard to solve in regions of low density.