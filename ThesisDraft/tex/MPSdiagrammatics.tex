\chapter{Diagrammatic Representation of Matrix Product States} \label{chap:diagrams}

Due to the many indices, equations describing contractions of matrix product states are often quite hard to read. Therefore, the equations are often represented graphically through diagrams. Many variations of diagrams exists, however, they all follow some general rules:
\begin{itemize}
\item
Tensors are represented by nodes.
\item
Indices are represented by lines. Horisontal lines are \textit{bond indices}, while vertical lines are \textit{physical indices}.
\item
Tensors connected by a line are contracted over said bond.
\end{itemize}
Through these basic rules, most equations involving matrix product states can be expressed through diagrams. In this instance, the color or shape of a node provides information regarding the properties of the tensor. Therefore, the tensors shown in this thesis follow these guidelines:
\begin{itemize}
\item
Tensors part of an MPS are square-like. The color of the tensor denotes its normalisation, as shown in figure \ref{fig:MPStensors}. These different types of normalisation are the essence of MPS canonical forms (see Section \ref{sec:canonical}).
\item
Tensors part of MPO's are grey and circular/elliptical depending on the number of sites they span over. These tensors have two vertical legs compared two the one leg of the MPS tensors.
\item
Matrices (tensors without any physical index) are depicted as a diamond. These often occur after a singular value decomposition. 
\end{itemize} 


\begin{figure}[h!]
\centering % <-- add this
\begin{subfigure}[t]{0.2\textwidth}
	\caption{}  	
  	\begin{tikzpicture}[inner sep=1mm]
	\node[tensor, label={\small $M^{[n]}$}] (tensor1) at (0, 0) {};
	\node (index1) at (0, -1) {$j_n$};
	\node (index2) at (1, 0) {$\alpha_{n}$};
	\node (index3) at (-1.2, 0) {$\alpha_{n-1}$};
	
	\draw[-] (tensor1) -- (index1);
	\draw[-] (tensor1) -- (index2);
	\draw[-] (tensor1) -- (index3);
\end{tikzpicture}
\end{subfigure}
\hspace{5mm}
\begin{subfigure}[t]{0.2\textwidth}    
	\caption{}  	
  	\begin{tikzpicture}[inner sep=1mm]
	\node[tensorl, label={\small $A^{[n]}$}] (tensor1) at (0, 0) {};
	\node (index1) at (0, -1) {};
	\node (index2) at (1, 0) {};
	\node (index3) at (-1, 0) {};
	
	\draw[-] (tensor1) -- (index1);
	\draw[-] (tensor1) -- (index2);
	\draw[-] (tensor1) -- (index3);
\end{tikzpicture}
\end{subfigure}
\hspace{5mm}
\begin{subfigure}[t]{0.2\textwidth}    
	\caption{}  	
  	\begin{tikzpicture}[inner sep=1mm]
	\node[tensorr, label={\small $B^{[n]}$}] (tensor1) at (0, 0) {};
	\node (index1) at (0, -1) {};
	\node (index2) at (1, 0) {};
	\node (index3) at (-1, 0) {};
	
	\draw[-] (tensor1) -- (index1);
	\draw[-] (tensor1) -- (index2);
	\draw[-] (tensor1) -- (index3);
\end{tikzpicture}
\end{subfigure}
\hspace{5mm}
\begin{subfigure}[t]{0.2\textwidth}    
	\caption{}  	
  	\begin{tikzpicture}[inner sep=1mm]
	\node[tensorc, label={\small $\Psi^{[n]}$}] (tensor1) at (0, 0) {};
	\node (index1) at (0, -1) {};
	\node (index2) at (1, 0) {};
	\node (index3) at (-1, 0) {};
	
	\draw[-] (tensor1) -- (index1);
	\draw[-] (tensor1) -- (index2);
	\draw[-] (tensor1) -- (index3);
\end{tikzpicture}
\end{subfigure}
\caption{\textit{The four different types of MPS tensors used in diagrams. \textbf{(i)} General tensor of un-specified normalisation. The indices corresponding to the tensor are labeled. The remaining tensors are: Left-normalised \textbf{(ii)}, right-normalised \textbf{(iii)}, central cite \textbf{(iv)}. }}
\label{fig:MPStensors}
\end{figure}
	
